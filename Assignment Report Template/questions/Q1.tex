\section{مروری بر نمونه برداری}

در این سوال قصد داریم صرفا مروری بر مفاهیم اولیه نمونه برداری داشته باشیم. در ابتدا ضرایب سری فوریه قطار ضربه ای با تناوب T را بدست
آورید سپس با استفاده از ضرایب سری فوریه آن تبدیل فوریه این سیگنال را بدست آورید.

\begin{qsolve}[]
	\begin{eqnarray*}
		p(t)&=&\sum_{n=-\infty}^{\infty}\delta(t-nT)\qquad \text{\lr{periodic in T}}\\
		a_k &=&\frac{1}{T}\int_{-T/2}^{T/2}f(t)e^{jk(2\pi/T)t}dt=
		\frac{1}{T}\int_{-T/2}^{T/2}\sum_{n=-\infty}^{\infty}\delta(t-nT)e^{jk(2\pi/T)t}dt
		\xrightarrow{n=0}\\
		&=&\frac{1}{T}\int_{-T/2}^{T/2}\delta(t)e^{jk(2\pi/T)t}dt=\frac{1}{T}
		\hspace{8em}\text{\hl{$a_k=\frac{1}{T}$}}
	\end{eqnarray*}
	تابع متناوب است و سری فوریه دارد، پس تبدیل فوریه آن همان سری فوریه آن به صورت
	جمع ضربه هاست.
	\begin{eqnarray*}
		P(j\omega)&=&\mathcal{F}\left\{\sum_{k=-\infty}^{\infty}a_ke^{jk(2\pi/T)t}\right\}=\sum_{k=-\infty}^{\infty}2\pi a_k\delta(\omega-k\frac{2\pi}{T})
		=\frac{2\pi}{T}\sum_{k=-\infty}^{\infty}\delta(\omega-k\frac{2\pi}{T})
	\end{eqnarray*}
\end{qsolve}

با توجه به اینکه با استفاده از این سیگنال نمونه برداری قطار ضربه از یک سیگنال پیوسته زمان انجام میدهیم و با استفاده نتیجه بدست آمده در قسمت
قبل تبدیل فوریه سیگنال بدست آمده حاصل از ضرب این سیگنال در سیگنال دلخواه $x(t)$ را بدست آورید.

\begin{qsolve}[]
	\begin{eqnarray*}
		x_p(t)&=&x(t)\times p(t)\Rightarrow X_p(j\omega)=\frac{1}{2\pi}X(j\omega)*P(j\omega)
		=\frac{1}{2\pi}\convolve[\omega]{P}{X}\\
		X_p(j\omega)&=&\frac{1}{2\pi}\sum_{k=-\infty}^{\infty}\intinf \frac{2\pi}{T}\delta(\tau-k\frac{2\pi}{T})X(j(\omega-\tau))d\tau
		=\frac{1}{T}\sum_{k=-\infty}^{\infty}X(j(\omega-k\frac{2\pi}{T}))
	\end{eqnarray*}
\end{qsolve}

حال نتیجه بدست آمده از قسمت قبل را با تبدیل فوریه گسسته سیگنال نمونه برداری شده $x[n]=x(nT)$ مقایسه کنید. حال شرطی روی نرخ
نمونه برداری $F_s=\frac{1}{T_s}$ بدست آورید که تمام محتوای فرکانسی سیگنال اولیه پس از نمونه برداری حفظ شود. (شرط نایکوییست)

\begin{qsolve}[]
	پاسخ بدست آمده متناوب است، پس میتوان آن را به صورت یک سری فوریه نوشت.
	یعنی داریم که : $X(j\omega)=\sum_{n=-\infty}^{\infty}C_ke^{jk\omega_0\omega}$
	\splitqsolve
	\begin{eqnarray*}
		X_p(j\omega)&=&\frac{1}{T}\sum_{k=-\infty}^{\infty}X(j(\omega-k\frac{2\pi}{T}))
		\xrightarrow{\text{\lr{periodic in $\frac{2\pi}{T}$}}}\sum_{n=-\infty}^{\infty}C_ne^{jnT\omega}\\
		C_n&=&\frac{1}{2\pi/T}\int_{-\pi/T}^{\pi/T}X_p(j\omega)e^{jnT\omega}d\omega
		=\frac{T}{2\pi}\int_{-\pi/T}^{\pi/T}\frac{1}{T}\sum_{k=-\infty}^{\infty}X(j(\omega-k\frac{2\pi}{T}))e^{jnT\omega}d\omega\\
		&=&\xrightarrow{k=0}\frac{1}{2\pi}\int_{-\pi/T}^{\pi/T}X(j\omega)e^{jnT\omega}d\omega
	\end{eqnarray*}
	حال اگر داشتیم که $X(j\omega)\{|\omega|\leq\frac{\pi}{T}\}=0$
	\begin{eqnarray*}
		C_n&=&\frac{1}{2\pi}\int_{-\pi/T}^{\pi/T}X(j\omega)e^{jnT\omega}d\omega=
		\frac{1}{2\pi}\int_{-\infty}^{\infty}X(j\omega)e^{jnT\omega}d\omega=x(nT)\\
		X_p(j\omega)&=&\sum_{n=-\infty}^{\infty}C_ne^{jnT\omega}=\sum_{n=-\infty}^{\infty}x(nT)e^{jnT\omega}
		=\sum_{n=-\infty}^{\infty}x[n]e^{jnT\omega}=X(e^{j\Omega})\when_{\Omega=\omega T}
	\end{eqnarray*}
	که این به ما شرط نایکوییست را میدهد. به صورتی که:
	\[
		\text{ : اگر داشته باشیم}
        \forall \omega \geq \omega_s = \frac{\pi}{T}: X(j\omega)=0\Longrightarrow X_p(j\omega)=X(e^{j\Omega})\when_{\Omega=\omega T}
	\]
\end{qsolve}

فرض کنید به علت محدودیت هایی که داریم شرط نایکوییست برقرار نباشد در این حالت روشی پیشنهاد دهید که محتوای فرکانسی کمتری از سیگنال
در اثر نمونه برداری از بین برود.

\begin{qsolve}[]
    ناچار ایم که مقداری اطلاعات از دست بدهیم، زیرا در حالت خالص نمونه برداری، شرط نایکوییست برقرار نیست 
    و دچار اعوجاج فرکانسی هستیم.

    میتوانیم قبل از نمونه برداری اطلاعات فرکانسی را کم کنیم، اینگونه اطلاعات از بین رفته و 
    فرکانسی اطلاعات کاهش میابد، ولی حداقل اعوجاج فرکانسی نداریم و نویز با فرکانس پایین ایجاد نمیکنیم.

    به این کار Anti-Aliasing میگوییم. به این صورت که قبل از نمونه برداری، یک \lr{low-pass filter} استفاده میکنیم.

    \begin{center}
        \includegraphics*[width=0.8\linewidth]{pics/anti-aliasing.png}
        \captionof{figure}{\lr{anti aliasing diagram}}
    \end{center}
\end{qsolve}