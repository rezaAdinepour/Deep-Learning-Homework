\section{مدولاسیون دامنه}

سیگنال پیغام $m(t)$ تبدیل فوریه ای به شکل زیر دارد:

\begin{figure}[h]
	\centering
	\includegraphics*[width=0.5\linewidth]{pics/q3_1.png}
\end{figure}

این سیگنال به عنوان ورودی به سیستم زیر داده میشود.

\begin{figure}[h]
	\centering
	\includegraphics*[width=0.6\linewidth]{pics/q3_2.png}
\end{figure}

الف) تبدیل فوریە ی سیگنال خروجی، $Y(f)$ را رسم کنید

ب) نشان دهید اگر خروجی این سیستم را به عنوان ورودی به همین سیستم بدهیم سیگنال پیام را میتوانیم بازیابی کنیم.

\begin{qsolve}[]
	مدولاسیون ناشی از ضرب در کسینوس را در نظر میگیریم:

	\begin{center}
		\includegraphics*[width=0.5\linewidth]{pics/cosine_mod.png}
		\captionof{figure}{مدولاسیون با کسینوس}
		\label{cosine_mod}
	\end{center}
	سپس فرایند را با نمودار توصیف میکنیم.
	\splitqsolve

	\begin{center}
		\includegraphics*[width=0.9\linewidth]{pics/q3_ans.png}
		\captionof{figure}{نتیجه مدولاسیون}
		\label{q3_mod}
	\end{center}
	که در مرحله آخر شکل \ref*{q3_mod} نتیجه یعنی همان $Y(f)$ را میبینیم.
\end{qsolve}

ب) نشان دهید اگر خروجی این سیستم را به عنوان ورودی به همین سیستم بدهیم سیگنال پیام را میتوانیم بازیابی کنیم.

\begin{qsolve}[]
	از فرایند توصیف شده در شکل \ref*{q3_mod} واضح است که سیستم طراخی شده، سیگنال ای
	با بیشینه فرکانس $W$ را میگیرد و سمت چپ و راست را در حوزه فرکانس جابجا میکند،
	به صورت معادله میتوان نوشت که:
	\[
		Y(f)=\begin{cases}
			M(-W+f) & 0\leq f \leq W   \\
			M(W+f)  & -W \leq f \leq 0 \\
			0       & \text{o.w.}
		\end{cases}
	\]

    بدیهی است که اگر این سیستم را دو بار اعمال کنیم، باز هم سمت راست و جپ سیگنال 
    در حوزه فرکانس جابجا شده و به سیگنال اصلی میرسیم.
\end{qsolve}