\section{سوال اول - شبکه‌های مولد تقابلی}
شبکه‌های مولد تقابلی\footnote{\lr{‫‪Networks‬‬‫‪Adverserial‬‬ ‫‪Generative‬‬}} همانطور که در کلاس با آنها آشنا شدید شامل دو زیرشبکه‌ی تولیدکننده\footnote{\lr{‫‪Generator‬‬}} و تمایزگر\footnote{\lr{‫‪Discriminator‬‬}} هستند که به صورت تقابلی آموزش داده می‌شوند تا داده‌های جدید تولید کنند. تولید جدید هدفی است که در تمامی مدل‌های مولد مد نظر قرار دارد و به شکل‌های مختلف از جمله ترجمه‌ی تصویر به تصویر، تبدیل دامنه و تولید شرطی صورت می‌گیرد. یکی از این اشکال، تولید تصویر با دریافت فرمان زبانی است که امروزه نیز نمونه‌های کاربردی آن همچون \lr{Imagen} و \lr{Dall-E} در دسترس عموم قرار دارند. در این تمرین به طور خاص به پیاده سازی این وظیفه با شبکه‌ی مولد تقابلی پشته‌ای یا \lr{SatckGAN} می‌پردازیم.



\begin{enumerate}
	\item
	به مراجعه به مقاله \href{https://arxiv.org/abs/1612.03242}{\textcolor{magenta}{\lr{StackGAN}}} کلیت ساختار و چگونگی عملکرد این شبکه را توضیح دهید. توضیح دهید که شبکه‌ی تعریف شده در هر گام\footnote{\lr{Stage}} به چه منظور استفاده می‌شود. به طور خاص ذکر کنید که ورودی شبکه‌ی تولیدکننده در هر دو گام چه تفاوتی با ورودی یک شبکه‌ی مولد تقابلی ساده\footnote{\lr{Vanilla GAN}} دارد؟ همچنین بررسی کنید که آموزش این شبکه به چه صورت انجام می‌شود.
	
	\begin{center}
		\includegraphics*[width=0.9\linewidth]{pics/img1.png}
		\captionof{figure}{معماری کلی شبکه مولد تقابلی پشته ای}
		\label{معماری کلی شبکه مولد تقابلی پشته ای}
	\end{center}
	
	\begin{qsolve}
		 
	\end{qsolve}
	
	
	
	
	\item 
شبکه‌های مولد تقابلی در مقایسه با سایر شبکه‌ها از سه مشکل اساسی رنج می‌برند: این سه مشکل عبارتند از فروپاشی مد\footnote{\lr{Mode Collapse}}، عدم همگرایی و ناپدید شدن گرادیان. به طور مختصر توضیح دهید که هر کدام به چه صورتی و چه راهکارهایی برای رفع آنها مطرح شده است؟

	\begin{qsolve}
		
	\end{qsolve}




	
	
	
	
	
	\item 
یک ایده‌ی رایج برای بهبود عملکرد شبکه‌های مولد تقابلی استفاده از عملکرد \lr{PixelShuffle} است. نحوه‌ی عملکرد این عملکرد و تأثیر آن را بررسی کنید. بررسی کنید که این عملکرد اولین بار در چه وظیفه‌ای و به چه منظور تعریف شد؟ همچنین بررسی کنید که به طور خاص در معماری \lr{StackGAN} در کدام زیرشبکه‌ها قابل استفاده است و چه عملکردی خواهد داشت؟

	\begin{qsolve}
		
	\end{qsolve}
	
	
	
	
	
	
	\item 
معیار \lr{FID (Frechet Inception Score)} یک معیار برای ارزیابی کیفیت و تنوع تصاویر تولید شده توسط مدل‌های مولد است. توضیح دهید که این معیار به چه صورت محاسبه می‌شود، به چه ویژگی‌هایی از مدل و یا داده وابسته است و آیا معیار قابل اتکایی برای مقایسه‌ی مدل‌های مولد محسوب می‌شود؟
	
	\begin{qsolve}
		
	\end{qsolve}
	
	
	
	
	
	\item 
مدل را بر روی این داده‌ها آموزش دهید. معماری نهایی هر یک از چهار زیرشبکه به همراه نمودار خطای تولیدکننده و تمایزگر در هر گام آموزش را در گزارش خود بیاورید. پس از پایان آموزش ۱۰ تصویر را به صورت تصادفی از خروجی مدل در \lr{stage} اول و دوم تولید کنید.
	
	
برای این پروژه از مجموعه داده‌ی \lr{CUB-2011} استفاده می‌کنیم که شامل یازده هزار تصویر از ۲۰۰ گونه پرنده می‌باشد و به ازای هر تصویر یک توصیف متنی نیز وجود دارد. \href{https://www.kaggle.com/datasets/veeralakrishna/200-bird-species-with-11788-images}{\textcolor{magenta}{مجموعه‌ی داده}} در سایت \lr{Kaggle} و توصیفات متنی نیز در \href{https://drive.google.com/file/d/0B3y_msrWZaXLT1BZdVdycDY5TEE/view?resourcekey=0-sZrhftoEfdvHq6MweAeCjA}{\textcolor{magenta}{این لینک}} موجود است. همچنین برای توصیفات متنی از پیش تعبیه‌ی آماده شده در فایل \lr{char-CNN-RNN-embeddings.pickle} وجود دارد که می‌تواند جایگزین ساختن داده باشد. استفاده از پیش تعبیه‌ها نیز که منجر به کارایی بهتر مدل شوند دارای ۵ امتیاز اضافی می‌باشد.
	
	\begin{center}
		\includegraphics*[width=0.9\linewidth]{pics/img2.png}
		\captionof{figure}{نمونه خروجی مدل \lr{StackGAN} برای مجموعه داده \lr{CUB}}
		\label{نمونه خروجی مدل StackGAN برای مجموعه داده CUB}
	\end{center}
	
	
	\begin{qsolve}
		
	\end{qsolve}
	
\end{enumerate}