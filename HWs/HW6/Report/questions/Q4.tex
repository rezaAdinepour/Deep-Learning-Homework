\section{سوال چهارم - نظری}

کانولوشن متسع\footnote{\lr{Dilated convolution}} روشی برای افزایش میدان پذیرش (\lr{Receptive field}) شبکه‌های کانولوشنی است که به‌صورت زیر تعریف می‌شود: (دقت شود خروجی تنها برای اندیس‌هایی از کرنل و تصویر همپوشانی کامل دارند، محاسبه می‌شود)

$$ (k \ast I)(i, j)=\sum_{-\infty}^{\infty}\sum_{-\infty}^{\infty} K(m,n)I(i+D_m, j+D_n) $$



\begin{enumerate}
	\item در یک شبکه کانولوشنی با یک لایه کانولوشن $K\times K$ با طول گام یک، عرض میدان پذیرش را بدست آورید.
	
	\begin{qsolve}
		
		
		 
		 
		 در لایه اول عرض میدان پذیرش برابر است با:
		 $$ R_1=K $$
		 
		 در لایه دوم داریم:
		 
		 $$ R_2=R_1+(K-1)=K+(K-1)=2K-1 $$
		 
		 برای سه لایه داریم:
		 $$ R_3=R_2+(K-1)=(2K-1)+(K-1)=3K-1$$
		 
		 و به‌صورت کلی فرمول عرض میدان پذیرش برای یک شبکه کانولوشنی با با \lr{L} لایه کانولوشن، به‌صورت زیر محاسبه می‌شود:
		 $$ R_L=R_{L-1} + (K - 1) $$
	\end{qsolve}
	
	
	
	
	\item برای ورودی 
	$I \in \mathbf{R}^{M\times N}$
	و کرنل
	$K \in \mathbf{R}^{F\times F}$،
		نشان دهید خروجی عملگر متسع دارای ابعاد 
	$(M-DF+D)\times (N-DF+D)$
	است. متغیر $D$ به معنی \lr{Dilation} است.
	
	
	
	\begin{qsolve}
		سایز موثر کرنل وقتی عملگر متسع اعمال شود به‌صورت زیر می‌شود:
		$$ F_{effective}=(F-1)\cdot D+1 $$
		
		تعداد موقعیت‌هایی که کرنل به‌صورت افقی می‌پیماید به‌صورت زیر است:
		$$ N-((F-1)\cdot D+1)+1=N-(F-1)\cdot D $$
		همچنین موقعیت‌های عمودی نیز به‌صورت زیر است:
		$$ M-((F-1)\cdot D+1)+1=M-(F-1)\cdot D $$
		
		بنابراین ابعاد نهایی به‌صورت زیر می‌شود:
		$$ (M-(F-1)\cdot D)×(N-(F-1)\cdot D)(M-(F-1)\cdot D)\times(N-(F-1)\cdot D) $$
		
	\end{qsolve}
	
	
	
	
	
	
	
	
	\item نشان دهید کانولوشن متسع معادل کانولوشن با کرنل متسع شده 
	$K'=K \otimes A$ 
	است. ماتریس $A$ را مشخص کنید. (عملگر $\otimes$ به معنی  \lr{Kronecker product} است.)
\end{enumerate}

















