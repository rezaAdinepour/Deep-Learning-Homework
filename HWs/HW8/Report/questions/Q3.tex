\section{سوال سوم - تئوری}
چرا و چگونه نمونه‌های خصمانه‌ی ایجاد شده برای یک مدل می‌توانند مدل‌های دیگر را نیز فریب دهند؟ این خاصیت انتقال‌پذیری چگونه می‌تواند در حملات جعبه سیاه استفاده شود؟


\begin{qsolve}
یک جنبه جالب و مهم از نمونه‌های خصمانه خاصیت انتقال‌پذیری آنهاست، یک نمونه خصمانه که برای فریب یک مدل طراحی شده است، می‌تواند مدل‌های دیگر را نیز فریب دهد. از این پدیده در حملات جعبه سیاه استفاده می‌شود، جایی که مهاجم دسترسی مستقیم به مدل هدف ندارد.

این موضوع می‌تواند به دلایل زیر باشد:

\begin{enumerate}
	\item \textbf{شباهت در ویژگی‌های آموخته شده}
	
	\item \textbf{داده‌های آموزشی مشترک}
	
	\item \textbf{آسیب‌پذیری‌های مشترک:}\\
مدل‌های مختلف، به ویژه آنهایی که بر روی همان مجموعه داده یا با معماری‌های مشابه آموزش دیده‌اند، تمایل به یادگیری ویژگی‌های مشابه دارند. مثال‌های خصمانه از ضعف‌های این ویژگی‌های آموخته شده سوءاستفاده می‌کنند، که باعث می‌شود به احتمال زیاد بین مدل‌ها انتقال یابند.

	\item \textbf{ماهیت خطی مدل‌ها:}\\
یسیاری از مدل‌های یادگیری ماشین، به ویژه شبکه‌های عصبی، طبیعتی خطی دارند. تغییرات خصمانه از این خطی بودن سوءاستفاده می‌کنند و از آنجا که این ویژگی بین مدل‌ها مشترک است، مثال‌های خصمانه می‌توانند انتقال پیدا کنند.

	\item \textbf{همپوشانی در مرز‌های تصمیم گیری:}\\
مدل‌هایی که بر روی داده‌های مشابه آموزش دیده‌اند، تمایل به داشتن مرزهای تصمیم‌گیری همپوشان دارند. مثال‌های خصمانه‌ای که برای عبور از مرز تصمیم‌گیری یک مدل ساخته شده‌اند، به احتمال زیاد از مرزهای تصمیم‌گیری مدل‌های دیگر که بر روی داده‌های مشابه آموزش دیده‌اند نیز عبور می‌کنند.
\end{enumerate}

در حملات جعبه سیاه، مهاجم دسترسی مستقیم به پارامترهای مدل هدف یا داده‌های آموزشی آن ندارد. در عوض، آنها از خاصیت انتقال‌پذیری مثال‌های خصمانه بهره می‌برند. به همین دلیل می‌توان مدلی جایگزین را آموزش داد که مدل هدف را تقریب بزند. این کار می‌تواند با استفاده از داده‌های مشابه یا تقلید از رفتار مدل هدف انجام شود.

هنگامی که مدل جایگزین آموزش دید، نمونه‌های خصمانه با استفاده از این مدل ساخته می‌شوند. سپس این نمونه‌های ساخته شده، به مدل هدف اعمال می‌شوند. به دلیل خاصیت انتقال‌پذیری، این نمونه‌ها احتمال بالایی برای فریب مدل هدف دارند.



\end{qsolve}