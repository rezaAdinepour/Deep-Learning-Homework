\section{سوال پنجم - تئوری}
چگونه می‌توان آموزش خصمانه را در مجموعه داده‌های نامتوازن پیاده‌سازی کرد و چه چالش‌هایی در این مسیر وجود دارد؟





\begin{qsolve}
آموزش خصمانه بر روی مجموعه‌داده‌های نامتوازن چالش‌های خاصی دارد و نیازمند توجه دقیق به منظور اطمینان از مقاومت در برابر حملات خصمانه و عملکرد خوب در کلاس‌های اقلیتی است. در ادامه به نحوه پیاده‌سازی آموزش خصمانه بر روی مجموعه‌داده‌های نامتوازن و چالش‌های خاص آن می‌پردازیم


\begin{enumerate}
	\item \textbf{پیاده‌سازی آموزش خصمانه بر روی مجموعه‌داده‌های نامتوازن}
	\begin{enumerate}
		\item ایجاد مثال‌های خصمانه:
		\begin{itemize}
			\item 
از تکنیک‌هایی مانند \lrP (روش گرادیان سریع)، \lr{PGD} (گرادیان پروژه‌شده) یا سایر الگوریتم‌های حمله خصمانه برای ایجاد مثال‌های خصمانه استفاده کنید.

			\item 
می‌بایست اطمینان حاصل شود که مثال‌های خصمانه برای هر دو کلاس اکثریت و اقلیت ایجاد می‌شوند تا مدل به سمت کلاس اکثریت منحرف نشود.
		\end{itemize}
		
		
		\item آموزش خصمانه متوازن:
		\begin{itemize}
			\item 
آموزش خصمانه با وزن‌دهی به کلاس‌ها: به مثال‌های خصمانه از کلاس‌های اقلیت وزن بیشتری اختصاص دهید تا تاثیر بیشتری بر فرآیند یادگیری مدل داشته باشند.


			\item 
افزایش تعداد نمونه‌ها: برای ایجاد تعادل در مجموعه‌داده‌های آموزش خصمانه، تعداد بیشتری از مثال‌های خصمانه برای کلاس‌های اقلیت ایجاد کنید. این کار می‌تواند با ایجاد چندین مثال خصمانه از هر نمونه کلاس اقلیت انجام شود.

	
			\item 
کاهش تعداد نمونه‌ها: تعداد مثال‌های خصمانه برای کلاس اکثریت را کاهش دهید تا مجموعه‌داده‌های آموزشی متوازن شوند. این کار باید با احتیاط انجام شود تا از دست دادن اطلاعات مهم کلاس اکثریت جلوگیری شود.
		\end{itemize}
		
		
		\item روش‌های ترکیبی:
		\begin{itemize}
			\item 
ترکیب تکنیک‌های افزایش و کاهش تعداد نمونه‌ها برای حفظ یک مجموعه‌داده آموزشی خصمانه متوازن.


			\item 
استفاده از تکنیک‌های افزایش داده‌ها برای ایجاد نمونه‌های مصنوعی برای کلاس‌های اقلیت و اطمینان از تنوع و جلوگیری از بیش‌برازش.
		\end{itemize}
		
		
		
		\item آموزش خصمانه تطبیقی:
		\begin{itemize}
			\item 
تنظیم پویا قدرت حمله خصمانه بر اساس توزیع کلاس‌ها. به عنوان مثال، استفاده از اختلالات قوی‌تر برای کلاس‌های اکثریت و اختلالات ملایم‌تر برای کلاس‌های اقلیت برای حفظ تعادل.
		\end{itemize}
		
		
		
		
		\item توابع زیان متوازن برای کلاس‌ها:
		\begin{itemize}
			\item 
پیاده‌سازی توابع زیانی که عدم توازن کلاس‌ها را در نظر می‌گیرند، مانند زیان کانونی که زیان اختصاص داده شده به مثال‌های به‌خوبی طبقه‌بندی شده را کاهش می‌دهد و تمرکز بیشتری به مثال‌های سخت و کلاس‌های اقلیت می‌دهد.


			\item 
استفاده از تکنیک‌های یادگیری حساس به هزینه که برای طبقه‌بندی نادرست نمونه‌های کلاس اقلیت جریمه‌های بالاتری در نظر می‌گیرند.
		\end{itemize}
	\end{enumerate}
\end{enumerate}
	
\end{qsolve}





\begin{qsolve}
	\item \textbf{چالش‌ها در آموزش خصمانه بر روی مجموعه‌داده‌های نامتوازن}
	\begin{enumerate}
		\item 
	\end{enumerate}
	
	
	
	\item \textbf{استراتژی‌های کاهش چالش‌ها}
	\begin{enumerate}
		\item 
	\end{enumerate}
\end{qsolve}