\section{سوال هفتم - عملی}
در این سوال تصمیم داریم تا برای تصاویر ایرانی یک مدل با معماری رمزگذار و رمزگشا برای وظیفه شرح تصویر\footnote{\lr{Image Captioning}} طراحی کنیم. مجموعه داده \texttt{persian\_image\_captioning.rar} در اختیار شما قرار گرفته است. این مجموعه داده حدود ۱۵۰۰ مقاله خبری به همراه تصاویر مرتبط آن است. این مقالات از سایت خبرگزاری تسنیم جمع‌آوری شده است. فایل \texttt{news.json} حاوی لیستی از اشیاء \texttt{json} که هرکدام دارای اطلاعات زیر هستند:

\begin{enumerate}
	\item \lr{title: }عنوان مقاله خبری
	\item \lr{Description: }شرح کوتاهی از مقاله
	\item \lr{Category: }دسته‌ای که مقاله به آن تعلق دارد
	\item \lr{Reporter: }نام خبرنگاری که این مطلب را منتشر کرده است
	\item \lr{Time: }تاریخ و ساعتی که مقاله در آن منتشر شده است
	\item \lr{Images: }لیستی از تصاویر مرتبط با مقاله (همه آنها را می‌توانید در پوشه \lr{images} پیدا کنید)
\end{enumerate}


عنوان هر مقاله را می‌توان به عنوان یک شرح (\lr{caption}) برای تصاویر مرتبط با آن مقاله، در نظر گرفت. همچنین می‌توانید با جایگزین کردن مترادف کلمات و همچنین، با روش‌های دلخواه برای تصاویر، داده افزایی\footnote{\lr{‫‪‬‬‫‪Data‬‬ Augmentation}} کنید. در نهایت مدلی آموزش دهید تا این وظیفه را انجام دهد. موارد زیر را در گزارش خود لحاظ و توضیح کامل دهید:

\begin{enumerate}
	\item پیش‌پردازشی که انجام داده‌اید.
	\item معماری مدل پیشنهادی خود را رسم کنید.
	\item تابع هزینه‌ای\footnote{Loss Function} که استفاده کردید.
	\item روش‌هایی که برای ارزیابی این وظیفه استفاده شده.
\end{enumerate}


	اسکریپتی بنویسید تا با دریافت مسیر یک پوشه، شرح تصاویر در آن پوشه را در یک فایل \texttt{.txt} بنویسد. پوشه تحت عنوان \texttt{selected\_images} در اختیار شما قرار گرفته است. مسیر این پوشه را به اسکریپت خود بدهید و خروجی آن را (شرح تصاویر) همراه با تصاویر مرتبط ارسال کنید. دقت کنید که اسکریپت نوشته شده توسط شما در روز تحویل پروژه توسط دیگر تصاویر بررسی خواهد شد. تصاویر این پوشه در زیر نشان داده شده است:
	
\begin{center}
	\includegraphics*[width=1\linewidth]{pics/img2.png}
	\captionof{figure}{تصاویر پوشه \texttt{selected\_images}}
	\label{تصاویر پوشه selected_images}
\end{center}


توجه فرمایید نمره این تمرین (۳۰ + ۳۰) امتیازی است. یعنی در صورتی که مراحل پیش‌پردازش، معماری مدل، صحت نهایی و به طور کلی روش حل مسئله، دارای خلاقیت و کیفیت مورد قبولی باشد، علاوه بر نمره اصلی تا ۳۰ امتیاز، نمره اضافی برای شما در نظر گرفته خواهد شد.




\begin{qsolve}

\end{qsolve}