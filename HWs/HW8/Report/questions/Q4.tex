\section{سوال چهارم - تئوری}
چگونه می‌توان حملات خصمانه را در حوزه‌هایی مانند پردازش زبان طبیعی پیاده‌سازی کرد؟ چه چالش‌های خاصی در این حوزه وجود دارد؟




\begin{qsolve}
حملات خصمانه در حوزه پردازش زبان طبیعی شامل ایجاد ورودی‌های متنی است که به طور عمدی باعث می‌شود مدل‌های \lr{NLP} خروجی‌های نادرست یا نامطلوب تولید کنند. این حملات می‌توانند طیف وسیعی از کاربرد‌های \lr{NLP} مانند تحلیل احساسات، ترجمه ماشینی و طبقه‌بندی متن را هدف قرار دهند. پیاده‌سازی حملات خصمانه در \lr{NLP} نسبت به حوزه‌های دیگر مانند پردازش تصویر، چالش‌های منحصر به فردی دارد. در ادامه، به نحوه پیاده‌سازی این حملات و چالش‌های خاص آنها می‌پردازیم:


\begin{enumerate}
	\item \textbf{پیاده‌سازی حملات خصمانه در \lr{NLP}}
	\begin{enumerate}
		\item \textbf{ایجاد اختلال در متن:}
		\begin{itemize}
			\item 
جایگزینی مترادف‌ها: جایگزینی کلمات با مترادف‌هایشان برای ایجاد تغییرات جزئی در ورودی بدون تغییر کلی معنای آن. به عنوان مثال، جایگزینی خوشحال با شاد.
			
			\item 
ایجاد اختلال در سطح کاراکتر: وارد کردن اشتباهات کوچک یا جابجایی کاراکترها، مانند تغییر سلام به سللام.
			
			\item 
افزودن/حذف کلمات: افزودن یا حذف کلمات برای تغییرات جزئی در ورودی. برای مثال، اضافه کردن کلمه نه برای تغییر احساس جمله.
			
			\item 
بازنویسی: بازنویسی جمله با حفظ معنای اصلی آن اما با تغییر ساختار و انتخاب کلمات.
		\end{itemize}
		
		
		
		\item \textbf{حملات مبتنی بر گرادیان:}
		\begin{itemize}
			\item 
روش \lr{HotFlip}: روشی که کاراکترها را در متن بر اساس گرادیان‌های مدل تغییر می‌دهد تا حداقل تغییراتی که تأثیر زیادی بر خروجی مدل دارد را پیدا کند.

			\item 
روش \lr{TextFooler}: تکنیکی که با استفاده از گرادیان‌ها، مهم‌ترین کلمات در متن را شناسایی و آنها را با مترادف‌ها یا کلمات مشابه معنایی جایگزین می‌کند.
		\end{itemize}
		
		
		
		
		\item \textbf{حملات جعبه سیاه:}
		\begin{itemize}
			\item 
الگوریتم‌های ژنتیک: استفاده از الگوریتم‌های تکاملی برای تغییر تدریجی متن و ایجاد مثال‌های خصمانه بدون نیاز به دسترسی به گرادیان‌های مدل.

			\item 
حملات مبتنی بر پرس‌وجو: ارسال پرس‌وجوهای متعدد به مدل و مشاهده خروجی‌ها برای ساختن مثال‌های خصمانه، با استفاده از قابلیت انتقال‌پذیری حملات از مدل‌های جایگزین.
		\end{itemize}
	\end{enumerate}
	
	
	
	
	
	
	
	
	\item \textbf{چالش‌های خاص در حملات خصمانه \lr{NLP}}
	\begin{enumerate}
		\item 
ماهیت گسسته متن:
برخلاف تصاویر که می‌توان مقادیر پیکسلی را به طور پیوسته تغییر داد، متن گسسته است و هر تغییری باید منجر به جملات معتبر و معنادار شود. این امر ایجاد مثال‌های خصمانه را پیچیده‌تر می‌کند.


		\item 
حفظ معنا:
اطمینان از اینکه تغییرات خصمانه معنای اصلی متن را تغییر نمی‌دهد بسیار مهم است. تغییرات باید برای خوانندگان انسانی غیرقابل تشخیص باشد در حالی که همچنان بر خروجی مدل تأثیر می‌گذارد.

		\item 
دستور زبان:
اختلالات باید دستور زبان و ساختار نحوی صحیح را حفظ کنند تا از ایجاد جملات بی‌معنی یا نادرست از نظر گرامری جلوگیری کنند. این برای واقعی و منطقی بودن متن خصمانه مهم است.
	\end{enumerate}
\end{enumerate}
\end{qsolve}



\begin{qsolve}
	\begin{enumerate}
		\item [3.]\textbf{نمونه‌هایی از تکنیک‌های حمله خصمانه در \lr{NLP}}
		\begin{enumerate}
			\item 
			\lr{HotFlip:}
از اطلاعات گرادیان یک مدل در سطح کاراکتر استفاده می‌کند تا موثرترین تغییرات کاراکتری را که می‌تواند پیش‌بینی مدل را تغییر دهد شناسایی کند.

			\item
			\lr{TextFooler:}
مهم‌ترین کلمات در متن را که بر خروجی مدل تأثیر می‌گذارند شناسایی و آنها را با مترادف‌ها یا کلمات مشابه معنایی جایگزین می‌کند.

			\item 
			\lr{PWWS} (احتمال وزن‌دار اهمیت کلمات):
کلماتی را برای جایگزینی انتخاب می‌کند که برای پیش‌بینی مدل اهمیت دارند و احتمال حفظ معنای اصلی جمله بعد از جایگزینی را دارند.
		\end{enumerate}		
		
		
		
		
		
		\item [4.]\textbf{کاربردهای عملی و استراتژی‌های مقابله}
		\begin{enumerate}
			\item \textbf{کاربردها:}
			\begin{itemize}
				\item 
تحلیل احساسات: گمراه کردن مدل‌های تحلیل احساسات برای طبقه‌بندی اشتباه نظرات مثبت به عنوان منفی یا بالعکس.

				\item 
تشخیص هرزنامه: ایجاد پیام‌های هرزنامه که از فیلترهای هرزنامه عبور کنند.

				\item 
ترجمه ماشینی: وارد کردن خطاهای جزئی در ترجمه‌ها.
			\end{itemize}
			
			
			
			\item \textbf{استراتژی‌های مقابله:}
			\begin{itemize}
				\item 
آموزش خصمانه: آموزش مدل‌ها با مثال‌های خصمانه برای بهبود مقاومت.


				\item 
تقطیر دفاعی: کاهش حساسیت مدل به تغییرات جزئی با فشرده‌سازی دانش مدل به شکل ساده‌تر.


				\item 
پیش‌پردازش ورودی‌ها: نرمال‌سازی ورودی‌های متنی برای حذف احتمالی اختلالات قبل از رسیدن به مدل.
			\end{itemize}
		\end{enumerate}
	\end{enumerate}
\end{qsolve}







