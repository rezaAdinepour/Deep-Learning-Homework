\section{سوال ششم - عملی}
در این سوال می‌خواهیم یک حمله خصمانه با روش‌های \lr{FGSM} طراحی کنیم و سپس مدل از پیش آموزش داده شده \lr{ResNet18} را با آموزش خصمانه مقاوم سازیم. به این منظور مراحل زیر را دنبال کنید:


\begin{enumerate}
	\item 
مدل از پیش آموزش دیده \lr{ResNet18} را برای مجموعه داده \lr{CIFAR10} آموزش دهید. نمودار خطا آموزش و آزمون را رسم کنید.

	\item 
روش \lr{FGSM} را پیاده‌سازی کنید و 5 تصویر را به صورت تصادفی انتخاب کنید و به مدل حمله کنید. سپس برای این تصاویر، تصویر اصلی، تصویر آشفته شده\footnote{\lr{‫‪Perturbed‬‬}}، پرچسب اصلی و پرچسب پیش‌بینی شده بر روی تصویر آشفته شده را نمایش دهید.


	\item 
حال با گنجاندن نمونه‌های خصمانه در فرآیند آموزش، مدل \lr{ResNet18} را دوباره آموزش دهید (آموزش خصمانه). این فرآیند به مدل کمک می‌کند تا در برابر حملات خصمانه مقاوم‌تر شود. نحوه آموزش را کامل شرح دهید. نمودارهای زیر را در کنار هم رسم و تفسیر کنید.

	\begin{itemize}
		\item 
		\lr{train-natural}: خطای آموزش برروی مدل طبیعی
		
		\item 
		\lr{train-adversary}: خطای آموزش برروی مدل خصمانه
		
		\item 
		\lr{test-natural}: خطای آموزش برروی مدل طبیعی (مجموعه داده آزمون بدون تغییر)
		
		\item 
		\lr{test-adversary}: خطای آموزش برروی مدل خصمانه (مجموعه داده آزمون بدون تغییر)
	\end{itemize}
	
	
	\item 
تا اینجا ما توانستیم تا با حملات خصمانه تصویری که تفاوت بسیار کمی با دیتای اصلی دارد، مدل را به اشتباه بیندازیم. حال می‌خواهیم به صورت هدفمند اینکار را انجام دهیم؛ یعنی مدل باید به اشتباه کلاس مورد نظر ما را پیش‌بینی کند\footnote{\lr{Target Attack}}. با روش \lr{FGSM} حمله هدفمند را پیاده‌سازی و نحوه انجام آن را بطور کامل شرح دهید. حال با ایجاد نمونه‌های خصمانه جدید از مجموعه داده آزمون و همچنین داده‌های آزمون بدون تغییر، صحت هر دو مدل را (مدل طبیعی و مدل آموزش دیده به صورت خصمانه) را ارزیابی کنید. نتایج را تفسیر کنید. در مورد اثربخشی آموزش خصمانه در بهبود استحکام مدل در برابر حملات خصمانه بحث کنید.
\end{enumerate}





\begin{qsolve}
در ابتدا می‌بایست دیتاست \texttt{CIFAR10} را دانلود نموده. این دیتاست را دانلود کردیم و ۵ تصویر از ۱۰ کلاس موجود در این دیتاست را به صورت زیر نمایش دادیم:
\end{qsolve}

\begin{qsolve}
	\begin{center}
		\includegraphics*[width=1\linewidth]{pics/img3.png}
		\captionof{figure}{تصاویری رندوم از دیتاست \texttt{CIFAR10}}
		\label{تصاویری رندوم از دیتاست CIFAR10}
	\end{center}
\end{qsolve}

\begin{qsolve}
سپس وزن‌های شبکه از‌پیش آموزش داده شده \texttt{ResNet18} را لود می‌کنیم. چون شبکه \texttt{ResNet} بر روی دیتاست \texttt{ImageNet} آموزش دیده است، نیاز است که حتما آن را در چند \lr{Epoch} محدود با دیتاست خودمان آموزش مجدد بدهیم (\lr{Fine-tune} کنیم).

این کار را در ۵۰ ایپاک انجام داده‌ایم و دقت و خطای آموزش شبکه به صورت زیر شده است:

\begin{center}
	\includegraphics*[width=0.5\linewidth]{pics/img5.png}
	\captionof{figure}{روند تغییر دقت و خطای آموزش}
	\label{روند تغییر دقت و خطای آموزش}
\end{center}

همچنین نمودار خطا و دقت آموزش به صورت زیر بدست آمده است:

\begin{center}
	\includegraphics*[width=1\linewidth]{pics/img4.png}
	\captionof{figure}{نمودار خطا و دقت آموزش}
	\label{نمودار خطا و دقت آموزش}
\end{center}
\end{qsolve}




\begin{qsolve}
	همچنین دقت شبکه برای داده‌های \lr{test} نیز به صوزت زیر به‌دست آمده است:
	
\begin{latin}
	\texttt{Test Accuracy : 74.7\%}
\end{latin}

در مرحله بعد، حمله خصمانه \lr{FGSM} را با استفاده از کتابخانه \texttt{torchattacks} به ازای مقدار $\epsilon$ های متفاوت انجام می‌دهیم و مقدار دقت شبکه را به ازای هر $\epsilon$ نمایش می‌دهیم:

\begin{center}
	\includegraphics*[width=0.7\linewidth]{pics/img6.png}
	\captionof{figure}{روند تغییر دقت مدل به ازای $\epsilon$ های متفاوت}
	\label{روند تغییر دقت شبکه به ازای اپسیلون های متفاوت}
\end{center}


همچنین نمودار این تغییرات نیز به‌صورت زیر شده است:
\begin{center}
	\includegraphics*[width=0.8\linewidth]{pics/img7.png}
	\captionof{figure}{نمودار تغییرات دقت مدل به ازای $\epsilon$ های متفاوت}
	\label{نمودار تغییرات دقت شبکه به ازای اپسیلون های متفاوت}
\end{center}

و سپس شبکه را تست می‌کنیم. به صورت رندوم تعدادی از تصاویر را انتخاب می‌کنیم. تصاویر انتخاب شده و کلاس آنها به صورت زیر است:

\begin{center}
	\includegraphics*[width=0.8\linewidth]{pics/img8.png}
	\captionof{figure}{تصاویر رندوم انتخاب شده}
	\label{تصاویر رندوم انتخاب شده}
\end{center}

\end{qsolve}




\begin{qsolve}
	
	
	خروجی شبکه متخاصم به ازای $\epsilon=2/255$
	\begin{center}
		\includegraphics*[width=0.8\linewidth]{pics/img9.png}
		\captionof{figure}{خروجی شبکه متخاصم به ازای $\epsilon=2/255$}
		\label{خروجی شبکه متخاصم به ازای epsilon=2/255}
	\end{center}
	
	
	خروجی شبکه متخاصم به ازای $\epsilon=4/255$
	\begin{center}
		\includegraphics*[width=0.8\linewidth]{pics/img10.png}
		\captionof{figure}{خروجی شبکه متخاصم به ازای $\epsilon=4/255$}
		\label{خروجی شبکه متخاصم به ازای epsilon=4/255}
	\end{center}
	
	
	خروجی شبکه متخاصم به ازای $\epsilon=6/255$
	
	\begin{center}
		\includegraphics*[width=0.8\linewidth]{pics/img11.png}
		\captionof{figure}{خروجی شبکه متخاصم به ازای $\epsilon=6/255$}
		\label{خروجی شبکه متخاصم به ازای epsilon=6/255}
	\end{center}
	
	مشاهده می‌شود که شبکه در تشخصی به اشتباه افتاده از و تعدادی از نمونه ها را اشتباه تشخصی داده است. برای حل این موضوع، شبکه را مجددا آموزش می‌دهیم. اما این بار آموزش خصمانه.
	
	شبکه را در ۱۰۰ دوره آموزش می‌دهیم. روند تغییرات خطا و دقت به صورت زیر به‌دست آمده است:
\end{qsolve}



\begin{qsolve}
	\begin{center}
		\includegraphics*[width=0.5\linewidth]{pics/img12.png}
		\captionof{figure}{روند تغییرات خطا و دقت شبکه در آموزش خصمانه}
		\label{روند تغییرات خطا و دقت شبکه در آموزش خصمانه}
	\end{center}
	
	همچنین نمودار خروجی نیز به‌صورت زیر است:
	
	\begin{center}
		\includegraphics*[width=1\linewidth]{pics/img13.png}
		\captionof{figure}{نمودار خطا و دقت شبکه در آموزش خصمانه}
		\label{نمودار خطا و دقت شبکه در آموزش خصمانه}
	\end{center}
	
	مجددا شبکه را به ازای مقادیر قبلی $\epsilon$ تست می‌کنیم. مقادیر بدست آمده به صورت زیر است:
\end{qsolve}



\begin{qsolve}
	\begin{center}
		\includegraphics*[width=0.7\linewidth]{pics/img14.png}
		\captionof{figure}{تست نمونه ها با شبکه آموزش داده شده جدید به ازای $\epsilon$ های متفاوت}
		\label{تست نمونه ها با شبکه آموزش داده شده جدید به ازای epsilon های متفاوت}
	\end{center}
	
	نمودار تغییرات دقت به ازای $\epsilon$ های متفاوت نیز به صورت زیر به‌دست می‌آید:
	
	\begin{center}
		\includegraphics*[width=0.7\linewidth]{pics/img15.png}
		\captionof{figure}{نمودار تغییرات دقت شبکه به ازای $\epsilon$ های متفاوت}
		\label{نمودار تغییرات دقت شبکه به ازای epsilon های متفاوت}
	\end{center}
	
	
	با مقایسه نغییرات در دو مرحله متوجه می‌شویم که با انجام آموزش خصمانه دقت شبکه در به‌دست آوردن خروجی های دستکاری شده افزایش یافته است. برای مثال به زای $\epsilon=8/255$، 
	۱۲٫۵ درصد افزایش دقت داشته ایم.
	
	
\end{qsolve}