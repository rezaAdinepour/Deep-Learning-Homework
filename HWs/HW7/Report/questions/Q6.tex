\section{سوال دوم - عملی}

مجموعه‌داده \lr{CoLA} (\lr{Corpus of Linguistic Acceptability}) یک مجموعه داده مهم در زمینه پردازش زبان طبیعی (\lr{NLP}) است که برای ارزیابی مقبولیت زبانی جملات استفاده می‌شود. مقبولیت زبانی به این معنی است که آیا یک جمله از نظر دستوری و نحوی توسط گویش‌وران بومی یک زبان درست است یا نه. در این سوال قصد داریم تا با تنظیم دقیق مدل \lr{BERT}، یک طبقه‌بند دو کلاسه برای این مجموعه‌داده پیاده‌سازی کنیم. موارد زیر را دنبال کنید:



\begin{enumerate}
	\item 
	دو فایل \texttt{in\_domain\_train.tsv} و \texttt{out\_of\_domain\_dev.tsv} در اختیار شما قرار گرفته است. این فایل‌ها را در محیط برنامه‌نویسی خود بارگزاری کنید. پیش پردازش‌های لازم (مانند اضافه کردن کارکترهای خاص \lr{[SEP]} و ...) به جملات، توکنایز کردن و ...
	
	
	\item 
۱۰ درصد از داده‌های \texttt{"in\_domain\_train.tsv"} را به برای اعتبارسنجی در نظر بگیرید.
	
	
	
	\item 
مدل \lr{BERT} را بارگذاری و پیکره‌بندی کنید. (پیشنهاد می‌شود از کتابخانه \lr{transformers}) استفاده کنید.



	\item 
مدل را آموزش دهید. در هر \lr{epoch}، خطا و صحت را برای داده‌های اعتبارسنجی چاپ کنید. همچنین بعد از اتمام آموزش نمودار خطا را به ازای هر دسته (\lr{batch}) آموزش رسم کنید. (هر \lr{epoch} می‌تواند شامل چندین دسته باشد).
	
	
	\item 
از داده‌های \texttt{out\_of\_domain\_dev.tsv} برای ارزیابی مدل تنظیم-دقیق شده خود استفاده کنید. برای این قسمت از معیار \lr{F1} و \lr{MCC1} استفاده کنید. این معیار را توضیح دهید و بگویید چرا استفاده از این معیار در اینجا نسبت به \lr{F1} بهتر است.

	
	\item 
معیار \lr{MCC} شما برای داده‌های \texttt{out\_of\_domain\_dev.tsv} نباید کوچکتر از ۰.۵ باشد.

\end{enumerate}







\begin{qsolve}
	ابتدا دیتاست را لود می‌کنیم. اندازه داده‌های آموزش $(8551, 4)$ است:
	
	\begin{center}
		\includegraphics*[width=0.9\linewidth]{pics/img11.png}
		\captionof{figure}{داده‌های آموزش دیتاست \texttt{in\_domain\_train.tsv}}
		\label{داده‌های آموزش دومین}
	\end{center}	
\end{qsolve}


\begin{qsolve}
		
	همانند سوال قبل، پیش‌پردازش های لازم مانند \lr{Tokenize} کردن و حذف کلمات \lr{Stop}، و اضاف نمودن \texttt{[CLS]} و \texttt{[SEP]} به توکن ها را انجام می‌دهیم.
	
	برای مثال، خروجی \lr{Tokenize} شده یکی از ورودی ها به‌صورت زیر است:
	
	\begin{center}
		\includegraphics*[width=0.9\linewidth]{pics/img12.png}
		\captionof{figure}{خروجی \lr{Tokenize} شده}
		\label{خروجی توکنایز شده}
	\end{center}
	
	برای آموزش شبکه از مدل از پیش آموزش دیده \texttt{Bert} استفاده می‌کنیم. مدل به صورت زیر تعرریف شده است:
	
	\begin{latin}
		\texttt{model = BertForSequenceClassification.from\_pretrained(}\\
		\texttt{"bert-base-uncased",}\\
		\texttt{num\_labels = 2,}\\
		\texttt{output\_attentions = False,}\\
		\texttt{output\_hidden\_states = False, )}
	\end{latin}
	
	\begin{center}
		\includegraphics*[width=0.9\linewidth]{pics/img13.png}
		\captionof{figure}{معماری مدل}
		\label{معماری مدل}
	\end{center}
\end{qsolve}



\begin{qsolve}
	همچنین پارامتر‌های شبکه به‌صورت زیر است:
	
	\begin{center}
		\includegraphics*[width=0.7\linewidth]{pics/img14.png}
		\captionof{figure}{پارامتر‌های شبکه}
		\label{پارامتر‌های شبکه}
	\end{center}
	
	و درنهایت شبکه را \lr{Fine tune} می‌کنیم. از آنجایی که مدل از پیش آموزش دیده است، نیازی به آموزش طولانی نیست. در حد ۵ الی ۱۰ دوره برای آموزش مناسب می‌باشد. در اینجا ما تعداد دوره های آموزشی را ۱۰ انتخاب کردیم.
	
	در نهایت پس از اتمام آموزش، مقدار خطا و دقت به‌صورت زیر به‌دست می‌آید:
	
	\begin{center}
		\includegraphics*[width=0.5\linewidth]{pics/img16.png}
		\captionof{figure}{مقدار دقت و خطا در پایان آموزش}
		\label{مقدار دقت و خطا در پایان آموزش}
	\end{center}
\end{qsolve}



\begin{qsolve}
	نمودار خطای آموزش برحسب تعداد \lr{Epoch} نیز به‌صورت زیر بدست می‌آید:
	
	\begin{center}
		\includegraphics*[width=0.8\linewidth]{pics/img15.png}
		\captionof{figure}{نمودار خطای آموزش}
		\label{نمودار خطای آموزش}
	\end{center}
	
	
	
\end{qsolve}