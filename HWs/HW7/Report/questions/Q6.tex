\section{سوال دوم - عملی}

مجموعه‌داده \lr{CoLA} (\lr{Corpus of Linguistic Acceptability}) یک مجموعه داده مهم در زمینه پردازش زبان طبیعی (\lr{NLP}) است که برای ارزیابی مقبولیت زبانی جملات استفاده می‌شود. مقبولیت زبانی به این معنی است که آیا یک جمله از نظر دستوری و نحوی توسط گویش‌وران بومی یک زبان درست است یا نه. در این سوال قصد داریم تا با تنظیم دقیق مدل \lr{BERT}، یک طبقه‌بند دو کلاسه برای این مجموعه‌داده پیاده‌سازی کنیم. موارد زیر را دنبال کنید:



\begin{enumerate}
	\item 
	دو فایل \texttt{in\_domain\_train.tsv} و \texttt{out\_of\_domain\_dev.tsv} در اختیار شما قرار گرفته است. این فایل‌ها را در محیط برنامه‌نویسی خود بارگزاری کنید. پیش پردازش‌های لازم (مانند اضافه کردن کارکترهای خاص \lr{[SEP]} و ...) به جملات، توکنایز کردن و ...
	
	\begin{qsolve}
		
	\end{qsolve}
	
	
	
	
	\item 
۱۰ درصد از داده‌های \texttt{"in\_domain\_train.tsv"} را به برای اعتبارسنجی در نظر بگیرید.
	\begin{qsolve}
		
	\end{qsolve}
	
	
	
	\item 
مدل \lr{BERT} را بارگذاری و پیکره‌بندی کنید. (پیشنهاد می‌شود از کتابخانه \lr{transformers}) استفاده کنید.
	\begin{qsolve}
		
	\end{qsolve}





	\item 
مدل را آموزش دهید. در هر \lr{epoch}، خطا و صحت را برای داده‌های اعتبارسنجی چاپ کنید. همچنین بعد از اتمام آموزش نمودار خطا را به ازای هر دسته (\lr{batch}) آموزش رسم کنید. (هر \lr{epoch} می‌تواند شامل چندین دسته باشد).
	\begin{qsolve}
		
	\end{qsolve}
	
	
	
	
	
	\item 
از داده‌های \texttt{out\_of\_domain\_dev.tsv} برای ارزیابی مدل تنظیم-دقیق شده خود استفاده کنید. برای این قسمت از معیار \lr{F1} و \lr{MCC1} استفاده کنید. این معیار را توضیح دهید و بگویید چرا استفاده از این معیار در اینجا نسبت به \lr{F1} بهتر است.
	\begin{qsolve}
		
	\end{qsolve}
	
	
	
	
	
	
	\item 
معیار \lr{MCC} شما برای داده‌های \texttt{out\_of\_domain\_dev.tsv} نباید کوچکتر از ۰.۵ باشد.


	
\end{enumerate}