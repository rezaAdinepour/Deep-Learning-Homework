\section{سوال پنجم - عملی}
شبکه‌های کانولوشنی با توجه به توانایی آن‌ها در استخراج و یادگیری خودکار ویژگی‌ها، مقاومت نسبت به تغییرات و کارایی آن‌ها در مقابل پیچیدگی‌های وظیفه‌ی بازشناسی چهره، یک عنصر اساسی در اکثر اسن سیستم‌ها هستند. در این تمرین قصد داریم که با استفاده از شبکه‌های عصبی کانولوشنی به تحلیل احساسات چهره\footnote{\lr{Facial expression recognition}} و طبقه‌بندی آن‌ها از روی تصویر بپردازیم. مجموعه داده‌ی این تمرین شامل ۱۲۰۰ تصویر نمونه‌گیری شده از هر کلاس مجموعه \lr{\href{https://paperswithcode.com/dataset/affectnet}{AffectNet}} می‌باشد. مجموعه داده \lr{AffectNet} شامل ۴۵۰ هزار تصویر چهره با ۸ حالت مختلف می‌باشد که شکل \ref{نمونه‌هایی از مجموعه داده AffectNet} نمونه‌هایی از آن را نشان می‌دهد.



\begin{center}
	\includegraphics*[width=0.6\linewidth]{pics/img2.png}
	\captionof{figure}{نمونه‌هایی از مجموعه داده \lr{AffectNet}}
	\label{نمونه‌هایی از مجموعه داده AffectNet}
\end{center}



\begin{enumerate}
	\item 
\end{enumerate}


