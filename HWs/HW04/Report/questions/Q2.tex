\section{سوال دوم - نظری}


شبکه‌های عمیق از عدم تفسیر‌پذیری رنج می‌برند. تلاش برای حل این مشکل، دو ایده \lr{Deconvolutional} و \lr{Up-convolutional} مظرح شده است. بررسی کنید و توضیح دهید هرکدام از دو روش، به چه صورت منجر به تفسیرپذیری می‌شوند؟



\begin{qsolve}
	پیش از توضیح دادن این دو روش که چگونه به تفسیرپذیری کمک می‌کنند، ابتدا این دو روش را مختثرا توضیح می‌دهیم.
	
	
	\begin{enumerate}
		\item \textbf{شبکه \lr{Deconvolutional} یا \lr{Transposed convolutional layer}: }\\
در لایه‌های کانولوشن ویژگی‌های مهم تصویر با استفاده از یک کرنل استخراج می‌شود و خروجی به عنوان \lr{Feature map} شناخته می‌شود. ابعاد تصویر (ممکن است) کاهش یابد و اطلاعات مهم تصویر حفظ می‌شود.


	\begin{center}
		\includegraphics*[width=0.6\linewidth]{pics/img4.png}
		\captionof{figure}{لایه کانولوشن}
		\label{لایه کانولوشن}
	\end{center}
	
	
	
	لایه \lr{Deconvolution} دقیقا برعکس لایه‌های کانولوشنی عمل می‌کند. یعنی از یک \lr{Feature map} می‌توان به تصویر رسید.

	\end{enumerate}
	
	
	
\end{qsolve}