\section{سوال اول - نظری}

همانگونه که در کلاس درس آشنا شده‌اید، واحد پردازشی پرسپترون و آدالاین امکان دریافت ورودی، توان‌های متعدد آن و حاصل ضرب ورودی ها را داشته و می‌تواند مسئله دسته‌بندی خطی را حل نمایند. در این سوال، قصد بدست آوردن وزن‌های یک نرون پردازشی پرسپترونی را به صورت نطری و با محاسبات دستی داریم.




\begin{center}
	\includegraphics*[width=1\linewidth]{pics/img1.png}
	\captionof{figure}{مسئله مورد بحث}
	\label{مسئله مورد بحث در سوال۱}
\end{center}



\begin{enumerate}
	\item شکل \lr{a-1} را برای دسته بندی مسئله دودویی درنظر بگیرید. معماری نورون مورد نظر را توضیح داده و وزن‌های آن را بدست آورید.
	
	\begin{qsolve}
		به دلیل آنکه داده های این قسمت جداپذر خطی هستند، می‌توان برای طبقه بندی آنها، از شبکه پرسپترون تک لایه استفاده کرد.
		
		معماری این شبکه به صورت «شکل \ref{معماری شبکه پرسپترون تک لایه}» است.
		
		
		\begin{center}
			\includegraphics*[width=0.8\linewidth]{pics/img2.pdf}
			\captionof{figure}{معماری شبکه پرسپترون تک لایه}
			\label{معماری شبکه پرسپترون تک لایه}
		\end{center}
	\end{qsolve}
	
	\begin{qsolve}
				در این شبکه، ورودی/خروجی ها با مربع های نارنجی، نورون ها با دایره سبز و تابع فعال ساز با مربع آبی نشان داده شده است. تعداد دیتا ورودی شبکه ۲ است. $x_1$ و $x_2$. بایاس این شبکه با $x_0$ نشان داده شده است. وزن های شبکه نیز با $w$ نشان داده شده است. بنابر این بردار ورودی و وزن‌های شبکه به صورت زیر است:
		
		$$
			X=\begin{bmatrix}          
				x_0=1\\
				x_1\\
				x_2
				
				\end{bmatrix} \\\       W=\begin{bmatrix}          
												w_0\\
												w_1\\
												w_2
											
											\end{bmatrix}
		$$
		
		طبق تئوری شبکه‌های عصبی می‌دانیم خروجی نرون به صورت زیر محاسبه می‌شود: (در اینجا برای انجام محاسبات ساده، تابع فعال‌ساز درنظر گرفته نشده است)
		
		$$
			\hat{y}=W^T X = \sum_{i=0}^{2} {w_ix_i}=w_0x_0+w_1x_1+w_2x_2 \xrightarrow{x_0=1} w_0+w_1x_1+w_2x_2
		$$
		
		طبق شکل \lr{a-1} دو نقطه از خط جدا کننده دو کلاس را داریم. بنابراین می‌توان معادله خط را به صورت زیر نوشت.
		
		می‌دانیم معادله خط به صورت زیر تعریف می‌شود:
		$$
			y-y_0=m(x-x_0)
		$$
		
		که در آن $m$ شیب خط است و به صورت زیر $\frac{\Delta y}{\Delta x}$ تعریف می‌شود. با جاگذاری یک از نقاط در معادله خط، می‌توان معادله خط را بدست آورد.
		
		$$
			P_1=\begin{bmatrix}          
				2.5\\
				0
				
			\end{bmatrix} \\\       P_2=\begin{bmatrix}          
				0\\
				2.8
				
			\end{bmatrix}
		$$
		
		$$
			m=\frac{2.8-0}{0-2.5}=-1.12 \rightarrow y-0=-1.12(x-2.5) \rightarrow \boxed{y=-1.12x+2.8}
		$$
		
		حالا اگر معادله خروجی نورون را به صورت زیر مرتب کنیم، می‌توان از مقایسه با مقادله خط بدست آمده وزن‌های شبکه را تعیین کرد.
		
		\begin{eqnarray*}
			x_1=\frac{-w_2}{w_1}x_2-\frac{w_0}{w_1},
			&x_2=\frac{-w_1}{w_2}x_1-\frac{w_0}{w_2}&
		\end{eqnarray*}

		
		در اینجا به دلیل آنکه دو معادله و ۳ مجهول ($w_0,w_1,w_2$) داریم، نیاز است که یکی از وزن ها را فرض کرده و دو وزن دیگر را بدست آورد.
		
%		\begin{eqnarray*}
%
%			x_1=\frac{-w_2}{w_1}x_2-\frac{w_0}{w_1}\\
%			x_2=\frac{-w_1}{w_2}x_1-\frac{w_0}{w_2}\\
%
%			\rightarrow y=-1.12x+2.8\\
%			
%
%			\frac{-w_2}{w_1}=
%			% \frac{-w_2}{w_1}=-1.12 \rightarrow w_2=1.12w_1\\
%			% \frac{-w_0}{w_1}=2.8 \rightarrow w_0=-2.8w_1\\
%
%		\end{eqnarray*}


%	\begin{eqnarray*}
%			p(t)&=&\sum_{n=-\infty}^{\infty}\delta(t-nT)\qquad \text{\lr{periodic in T}}\\
%			a_k &=&\frac{1}{T}\int_{-T/2}^{T/2}f(t)e^{jk(2\pi/T)t}dt=
%			\frac{1}{T}\int_{-T/2}^{T/2}\sum_{n=-\infty}^{\infty}\delta(t-nT)e^{jk(2\pi/T)t}dt
%			\xrightarrow{n=0}\\
%			&=&\frac{1}{T}\int_{-T/2}^{T/2}\delta(t)e^{jk(2\pi/T)t}dt=\frac{1}{T}
%			\hspace{8em}\text{\hl{$a_k=\frac{1}{T}$}}
%		\end{eqnarray*}
		
		
		\begin{eqnarray*}
			\frac{-w_2}{w_1}=-1.12 \rightarrow w_2=1.12w_1\\
			\frac{-w_0}{w_1}=2.8 \rightarrow w_0=-2.8w_1\\
			\rightarrow
			\begin{cases}
				w_2-1.12w_1=0\\
				-2.8w_1-w_0=0 
			\end{cases}
			\text{assume} &w_0&=2.8 \rightarrow \text{\hl{$w_1=-1$}}, \text{\hl{$w_2=-1.12$}}
		\end{eqnarray*}
		ذکر این نکته الزامیست که این جواب، یکتا نمی‌باشد و برحسب اینکه مقدار $w_0$ را چه انتخاب کنیم، مقدار ۲ وزن دیگر متفاوت می‌شود.
	\end{qsolve}
	
	
	
	
	
	
	
	
	
	
	
	
	
	
	
	
	
	
	
	\item حال شکل \lr{b-1} را درنظر بگیرید. چرا مسئله جداپذیر خطی نیست؟ چگونه می‌توان آنر را در قالب حل چند مسئله خطی حل نمود؟ معماری پیشنهادی خودتان را رسم و وزن‌های موجود در آن را با انجام محاسبات بدست آورید. معماری شما می‌تواند حاصل از کنار هم چیدن و پشت هم چیدن یک یا چند نورون پرسپترونی باشد.
\end{enumerate}


\begin{qsolve}
	سلام
\end{qsolve}








%
%
%
%
%
%
%
%\begin{qsolve}[]
%	\begin{eqnarray*}
%		p(t)&=&\sum_{n=-\infty}^{\infty}\delta(t-nT)\qquad \text{\lr{periodic in T}}\\
%		a_k &=&\frac{1}{T}\int_{-T/2}^{T/2}f(t)e^{jk(2\pi/T)t}dt=
%		\frac{1}{T}\int_{-T/2}^{T/2}\sum_{n=-\infty}^{\infty}\delta(t-nT)e^{jk(2\pi/T)t}dt
%		\xrightarrow{n=0}\\
%		&=&\frac{1}{T}\int_{-T/2}^{T/2}\delta(t)e^{jk(2\pi/T)t}dt=\frac{1}{T}
%		\hspace{8em}\text{\hl{$a_k=\frac{1}{T}$}}
%	\end{eqnarray*}
%	تابع متناوب است و سری فوریه دارد، پس تبدیل فوریه آن همان سری فوریه آن به صورت
%	جمع ضربه هاست.
%	\begin{eqnarray*}
%		P(j\omega)&=&\mathcal{F}\left\{\sum_{k=-\infty}^{\infty}a_ke^{jk(2\pi/T)t}\right\}=\sum_{k=-\infty}^{\infty}2\pi a_k\delta(\omega-k\frac{2\pi}{T})
%		=\frac{2\pi}{T}\sum_{k=-\infty}^{\infty}\delta(\omega-k\frac{2\pi}{T})
%	\end{eqnarray*}
%\end{qsolve}
%
%با توجه به اینکه با استفاده از این سیگنال نمونه برداری قطار ضربه از یک سیگنال پیوسته زمان انجام میدهیم و با استفاده نتیجه بدست آمده در قسمت
%قبل تبدیل فوریه سیگنال بدست آمده حاصل از ضرب این سیگنال در سیگنال دلخواه $x(t)$ را بدست آورید.
%
%\begin{qsolve}[]
%	\begin{eqnarray*}
%		x_p(t)&=&x(t)\times p(t)\Rightarrow X_p(j\omega)=\frac{1}{2\pi}X(j\omega)*P(j\omega)
%		=\frac{1}{2\pi}\convolve[\omega]{P}{X}\\
%		X_p(j\omega)&=&\frac{1}{2\pi}\sum_{k=-\infty}^{\infty}\intinf \frac{2\pi}{T}\delta(\tau-k\frac{2\pi}{T})X(j(\omega-\tau))d\tau
%		=\frac{1}{T}\sum_{k=-\infty}^{\infty}X(j(\omega-k\frac{2\pi}{T}))
%	\end{eqnarray*}
%\end{qsolve}
%
%حال نتیجه بدست آمده از قسمت قبل را با تبدیل فوریه گسسته سیگنال نمونه برداری شده $x[n]=x(nT)$ مقایسه کنید. حال شرطی روی نرخ
%نمونه برداری $F_s=\frac{1}{T_s}$ بدست آورید که تمام محتوای فرکانسی سیگنال اولیه پس از نمونه برداری حفظ شود. (شرط نایکوییست)
%
%\begin{qsolve}[]
%	پاسخ بدست آمده متناوب است، پس میتوان آن را به صورت یک سری فوریه نوشت.
%	یعنی داریم که : $X(j\omega)=\sum_{n=-\infty}^{\infty}C_ke^{jk\omega_0\omega}$
%	\splitqsolve
%	\begin{eqnarray*}
%		X_p(j\omega)&=&\frac{1}{T}\sum_{k=-\infty}^{\infty}X(j(\omega-k\frac{2\pi}{T}))
%		\xrightarrow{\text{\lr{periodic in $\frac{2\pi}{T}$}}}\sum_{n=-\infty}^{\infty}C_ne^{jnT\omega}\\
%		C_n&=&\frac{1}{2\pi/T}\int_{-\pi/T}^{\pi/T}X_p(j\omega)e^{jnT\omega}d\omega
%		=\frac{T}{2\pi}\int_{-\pi/T}^{\pi/T}\frac{1}{T}\sum_{k=-\infty}^{\infty}X(j(\omega-k\frac{2\pi}{T}))e^{jnT\omega}d\omega\\
%		&=&\xrightarrow{k=0}\frac{1}{2\pi}\int_{-\pi/T}^{\pi/T}X(j\omega)e^{jnT\omega}d\omega
%	\end{eqnarray*}
%	حال اگر داشتیم که $X(j\omega)\{|\omega|\leq\frac{\pi}{T}\}=0$
%	\begin{eqnarray*}
%		C_n&=&\frac{1}{2\pi}\int_{-\pi/T}^{\pi/T}X(j\omega)e^{jnT\omega}d\omega=
%		\frac{1}{2\pi}\int_{-\infty}^{\infty}X(j\omega)e^{jnT\omega}d\omega=x(nT)\\
%		X_p(j\omega)&=&\sum_{n=-\infty}^{\infty}C_ne^{jnT\omega}=\sum_{n=-\infty}^{\infty}x(nT)e^{jnT\omega}
%		=\sum_{n=-\infty}^{\infty}x[n]e^{jnT\omega}=X(e^{j\Omega})\when_{\Omega=\omega T}
%	\end{eqnarray*}
%	که این به ما شرط نایکوییست را میدهد. به صورتی که:
%	\[
%		\text{ : اگر داشته باشیم}
%        \forall \omega \geq \omega_s = \frac{\pi}{T}: X(j\omega)=0\Longrightarrow X_p(j\omega)=X(e^{j\Omega})\when_{\Omega=\omega T}
%	\]
%\end{qsolve}
%
%فرض کنید به علت محدودیت هایی که داریم شرط نایکوییست برقرار نباشد در این حالت روشی پیشنهاد دهید که محتوای فرکانسی کمتری از سیگنال
%در اثر نمونه برداری از بین برود.
%
%\begin{qsolve}[]
%    ناچار ایم که مقداری اطلاعات از دست بدهیم، زیرا در حالت خالص نمونه برداری، شرط نایکوییست برقرار نیست 
%    و دچار اعوجاج فرکانسی هستیم.
%
%    میتوانیم قبل از نمونه برداری اطلاعات فرکانسی را کم کنیم، اینگونه اطلاعات از بین رفته و 
%    فرکانسی اطلاعات کاهش میابد، ولی حداقل اعوجاج فرکانسی نداریم و نویز با فرکانس پایین ایجاد نمیکنیم.
%
%    به این کار Anti-Aliasing میگوییم. به این صورت که قبل از نمونه برداری، یک \lr{low-pass filter} استفاده میکنیم.
%
%    \begin{center}
%        \includegraphics*[width=0.8\linewidth]{pics/anti-aliasing.png}
%        \captionof{figure}{\lr{anti aliasing diagram}}
%    \end{center}
%\end{qsolve}