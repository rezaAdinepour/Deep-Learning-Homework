\section{سوال دوم - عملی}
مجموعه داده ضمیمه شده را بارگزاری کرده و آن را نمایش دهید. تفکیک مجموعه داده را با نسبت ۱:۲:۷ به‌ترتیب برای آموزش، آزمون و اعتبارسنجی درنظر بگیرید.






































%سیستم زیر را در نظر بگیرید:
%
%\begin{figure}[h]
%	\centering
%	\includegraphics*[width=0.6\linewidth]{pics/q2_1.png}
%\end{figure}
%
%سیگنال ورودی $x_c(t)$ دارای تبدیل فوریە ی زیر است.
%
%\begin{figure}[h]
%	\centering
%	\includegraphics*[width=0.3\linewidth]{pics/q2_2.png}
%\end{figure}
%
%که در آن :
%
%\[
%	\Omega_0=2\pi 1000rad/s
%\]
%
%سیستم گسسته زمان یک فیلتر پایین گذر ایده آل با پاسخ فرکانسی زیر است.
%
%\[
%	H(e^{j\omega})=\begin{cases}
%		1 & |\omega|<\omega_c \\
%		0 & \text{otherwise}
%	\end{cases}
%\]
%
%الف) کمینه نرخ نمونه برداری که در آن aⅼiasing رخ نمیدهد چقدر است؟
%
%\begin{qsolve}[]
%    کمینه نرخ نمونه برداری از حد نایکوییست بدست میاید بدین صورت که:
%    \[
%        \omega_s=\frac{2\pi}{T_s}\geq\omega_{nq}=2\omega_{max}=2\Omega_0\Rightarrow
%        T_s\leq\frac{\pi}{\Omega_0}=\frac{1}{2000}s=0.5ms
%    \]
%\end{qsolve}
%
%ب) اگر $\omega_c=\pi/2$ کمینه نرخ نمونه برداری به طوری که $y_t(t)=x_c(t)$؟
%
%
%\begin{qsolve}[]
%    اگر شرط نایکوییست ارضا شود، خواهیم داشت که:
%    \[
%        X(e^{j\Omega})=X_p(j\omega)\when_{\Omega=\omega T}    
%    \]
%    حال میخواهیم بعد از گسسته شدن، فیلتر اثری روی سیگنال نگذارد، یعنی کل محتوای فرکانسی زیر 
%    $\omega_c$ باشد. آنگاه:
%    \[
%      \Omega_{max}=\omega_{max}T=\Omega_0T\leq\omega_c=\frac{\pi}{2}\Rightarrow T\leq\frac{\pi}{2\Omega_0}
%      =\frac{1}{4000}s=0.25ms  
%    \]
%    که این شرط محدودیت بیشتری از شرط نایکوییست اعمال میکند.
%\end{qsolve}
