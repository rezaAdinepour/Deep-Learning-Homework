\section{سوال دوم - عملی}
مجموعه داده ضمیمه شده را بارگزاری کرده و آن را نمایش دهید. تفکیک مجموعه داده را با نسبت ۱:۲:۷ به‌ترتیب برای آموزش، آزمون و اعتبارسنجی درنظر بگیرید.

\textbf{تمامی کدهای سوالات عملی پیوست شده است. همچنین می‌توانید کد‌ها را از گیتهاب بنده به لینک زیر مشاهده و بررسی کنید: }

\begin{latin}
	\texttt{\href{https://github.com/rezaAdinepour/Deep-Learning-Homework}{github.com/rezaAdinepour/Deep-Learning-Homework}} 
\end{latin}


\begin{qsolve}
	
	پس از load کردن دیتاست و تقسیم کردن آن به ۳ دسته آموزش، تست و اعتبار سنجی، آن را رسم کردیم. خروجی به صورت زیر است:
	
	\begin{center}
		\includegraphics*[width=1\linewidth]{pics/img7.png}
		\captionof{figure}{دیتاست مسئله}
		\label{دیتاست مسئله}
	\end{center}
	
	همانطور که مشاهده می‌شود، داده ها جداپذیر حطی نیستند پس نمی‌توان به صورت معمولی آن را با شبکه پرسپترون حل نمود. در این سوال تلاش خواهیم کرد با روش‌های معرفی شده در سوال ۱، به صورت عملی با شبکه پرسپترون، یک مسئله غیر خطی را حل نماییم.
\end{qsolve}



\begin{enumerate}
	\item با یک نورون پرسپترونی و صرفا بر اساس ویژگی های ورودی، وزن‌های نورون خود را با آموزش بدست آورید و دسته‌بندی را انجام دهید. و معیار های صحت\footnote{\lr{Accuracy}} و امتیاز F1 را به ازای هر دسته گزارش نمایید. همچنین در نهایت وزن‌های معماری‌تان را به همراه طرحواره آن گزارش کنید.
	
	
	
	\begin{qsolve}
		
	\end{qsolve}
\end{enumerate}







































%سیستم زیر را در نظر بگیرید:
%
%\begin{figure}[h]
%	\centering
%	\includegraphics*[width=0.6\linewidth]{pics/q2_1.png}
%\end{figure}
%
%سیگنال ورودی $x_c(t)$ دارای تبدیل فوریە ی زیر است.
%
%\begin{figure}[h]
%	\centering
%	\includegraphics*[width=0.3\linewidth]{pics/q2_2.png}
%\end{figure}
%
%که در آن :
%
%\[
%	\Omega_0=2\pi 1000rad/s
%\]
%
%سیستم گسسته زمان یک فیلتر پایین گذر ایده آل با پاسخ فرکانسی زیر است.
%
%\[
%	H(e^{j\omega})=\begin{cases}
%		1 & |\omega|<\omega_c \\
%		0 & \text{otherwise}
%	\end{cases}
%\]
%
%الف) کمینه نرخ نمونه برداری که در آن aⅼiasing رخ نمیدهد چقدر است؟
%
%\begin{qsolve}[]
%    کمینه نرخ نمونه برداری از حد نایکوییست بدست میاید بدین صورت که:
%    \[
%        \omega_s=\frac{2\pi}{T_s}\geq\omega_{nq}=2\omega_{max}=2\Omega_0\Rightarrow
%        T_s\leq\frac{\pi}{\Omega_0}=\frac{1}{2000}s=0.5ms
%    \]
%\end{qsolve}
%
%ب) اگر $\omega_c=\pi/2$ کمینه نرخ نمونه برداری به طوری که $y_t(t)=x_c(t)$؟
%
%
%\begin{qsolve}[]
%    اگر شرط نایکوییست ارضا شود، خواهیم داشت که:
%    \[
%        X(e^{j\Omega})=X_p(j\omega)\when_{\Omega=\omega T}    
%    \]
%    حال میخواهیم بعد از گسسته شدن، فیلتر اثری روی سیگنال نگذارد، یعنی کل محتوای فرکانسی زیر 
%    $\omega_c$ باشد. آنگاه:
%    \[
%      \Omega_{max}=\omega_{max}T=\Omega_0T\leq\omega_c=\frac{\pi}{2}\Rightarrow T\leq\frac{\pi}{2\Omega_0}
%      =\frac{1}{4000}s=0.25ms  
%    \]
%    که این شرط محدودیت بیشتری از شرط نایکوییست اعمال میکند.
%\end{qsolve}
