\section{سوال سوم - عملی}

فرض کنید می‌خواهیم تابع $f(x)=sin(x)+3x^{17}-5x^2$ را با یک نورون پرسپترونی تخمین و درصورت امکان محاسبه نماییم.

\begin{enumerate}
	\item با تحقیق و مطالعه کافی توضیح دهید چگونه می‌توان از سری تیلور برای تخمین توابع استفاده نمود؟
	
	\begin{qsolve}
		
	\end{qsolve}
	
	
	
	
	
	\item حال یک نرون پرسپترونی طراحی نمایید که بتواند تابع فوق را با جملات سری تیلور محاسبه کند (وزن‌های نورون را با محاسبات بدست‌آورده و در گزارش خود بیان کنید.)
	در یک نمودار، تابع و تخمین‌های آن را به ازای استفاده از جملات ۱ تا ۱۰ رسم کنید (خروجی ۱۱ منهنی خواهد بود). در یک جدول خطای حاصل از تقریب را به ازای استفاده از جملات مختلف با تابع MSE گزارش کنید.
\end{enumerate}

\textbf{توجه مهم:‌ ورودی نورون‌های طراحی شده‌تان صرفا بایستی توانی از ویژگی‌های اصلی یا حاصل ضرب توانی از ویژگی‌ها باشد و فرم دیگری قابل قبول نیست. برای مثال اگر یک ویژگی $x$ باشد، $sin(x)$ یا $e^x$ نمی‌تواند ورودی یک نورون باشد.}

\begin{qsolve}
	
\end{qsolve}