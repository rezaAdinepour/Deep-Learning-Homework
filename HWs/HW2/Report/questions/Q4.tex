\section{سوال چهارم - نظری}
تقارن در شبکه عصبی به چه معناست؟ آیا نیاز داریم این تقارن را بشکنیم؟ درصورتی که جواب شما مثبت است، کیس مورد نظر را طراحی کرده و توضیح دهید. بررسی کنید چه روش‌های برای شکستن تقارن وجود دارد. چند مورد نام ببرید.





\begin{qsolve}
 	شاید در ابتدا با مطرح کردن مسئله تقارن در شبکه های عصبی، تقارن هندسی شبکه به ذهن آید اما باید گفت که در کنار این تقارن هندسی، تقارن در داده ها و تقارن در الگوریتم نیز وجود دارد. \cite{ref1}.
 	
 	تقارن در معماری شبکه بدین معنی است که تعداد نرون های ورودی و خروجی و نرون های مخفی نسبت به هم متقارن باشند. 
 	
 	تقارن در الگوریتم ها نیز یک نقش اساسی در یادگیری ماشین دارد. از آنجایی که الگوریتم های یادگیری ماشین معمولاً پیچیده هستند، استفاده از تقارن در آنها می تواند به تسهیل و بهبود عملکرد آنها کمک کند. به عنوان مثال، الگوریتم های شبکه های عصبی که از تقارن استفاده می کنند، می توانند سریع تر و دقیق تر اطلاعات را پردازش کنند و الگوهای پیچیده تری را متوجه شوند.
 	
 	تقارن در داده ها این مزیت را دارد که با افزایش دقت و سرعت در تحلیل داده ها و جلوگیری از بروز اشتباهات، در یادگیری کمک می کند. به عنوان مثال، اگر یک تصویر نیمه از یک چهره انسان در دیتاست باشد، تقارن در داده ها می تواند به یادگیری شبکه در تشخیص چهره انسان کمک کند. یادگیری از داده های تقارنی می تواند به شبکه کمک کند تا الگوهای پیچیده تر را بشناسند و بدون تحلیل دقیق تصاویر، به تصمیمات صحیح برسند.
 	
 	
همانند همه پدیده های حاکم بر طبیعت که متقارن هستند، در شبکه‌های عصبی مصنوعی نیز تقارن یک تصل مهم در طراحی است. روش هایی برای شکستن تقارن وجود دارد مانند:

\begin{enumerate}
	\item :Dropout در هر مرحله از آموزش به صورت رندم تعدادی از وزن ها را حذف می‌کند
	
	\item normalization Batch
	
	\item Augmentation :data  با اعمال تبدیلات مختلف به داده‌ها، مانند چرخش، برش، شیفت دادن و تغییر مقیاس، تنوع در داده‌ها افزایش می‌یابد و تقارن‌ها کاهش می‌یابند.
\end{enumerate}
 و با دلایل مطرح شده شکستن یا عدم شکستن تقارن در شبکه، بستگی به کاربرد و نوع تقارن دارد. در برخی از کاربرد ها می‌تواند مفید باشد و در برخی خیر.


\end{qsolve}

















\begin{latin}
	\begin{thebibliography}{9}
		\bibitem{ref1}
		Tanaka H, Kunin D. Noether’s learning dynamics: Role of symmetry breaking in neural networks. Advances in Neural Information Processing Systems. 2021 Dec 6;34:25646-60.
		
		\bibitem{ref2}
		Kaba SO, Ravanbakhsh S. Symmetry Breaking and Equivariant Neural Networks. arXiv preprint arXiv:2312.09016. 2023 Dec 14.
	\end{thebibliography} 
\end{latin}