\section{سوال پنجم - عملی}
ابروضوح یک کاربرد در بینایی کامپیوتر می‌باشد که در آن هدف ارتقای وضوح تصاویر می‌باشد. این امر می‌تواند در مقاصد مختلف نظیر تصویر برداری پزشکی، بهبود تصاویر نظارتی-امنیتی، بازسازی تصاویر قدیمی و ... به‌کار گرفته شود. در این سوال هدف طراحی و پیاده سازی یک شبکه عصبی چند‌لایه برای هدف فوق می‌باشد.



\begin{enumerate}
	\item ۱۰ تصویر دلخواه از اینترنت که حاوی گستره رنگی مختلفی می‌باشد را به‌عنوان مجموعه داده انتخاب کنید و آن را نمایش دهید. حال وضوح هر یک از تصاویر را نصف کنید. اکنون به ازای هر یک از پیکسل ها در عکس اصلی، متناظر آن و هشت همسایگی مجاور آن در عکس با وضوح پایین تر را بیابید و مجموعه داده موردنظر را بدست آورید. ابعاد ورودی برابر با ۲۷ ویژگی (پیکسل متناظر و هشت همسایگی آن به ازای سه کانال رنگی در وضوح پایین) خواهد بود و خروجی (لیبل) نیز شامل سه مقدار(مقدار سه کانال \lr{RGB} در تصویر اصلی) خواهد بود. 	این روند را برای تمامی پیکسل های ۱۰ تصویر انجام دهید تا برای هر تصویر $i$ یک مجموعه داده به صورت 
	$W_i*H_i, 27, 3$ پدید آید. دو تصویر را برای آزمون، و یک تصویر را برای اعتبار سنجی و هفت تصویر باقی مانده را برای آموزش استفاده کنید. می‌توانید پیکسل های حاصل از تصاویر مختلف در گروه آموزش را باهم ترکیب کرده و درهم (\lr{shuffle}) سازید که ابعاد آن مجموعه داده به صورت $\sum_{i=1}^{7} W_i*H_i, 27, 3$ در‌آید.
	


	
	
	
	
	
	\item یک شبکه چند لایه پرسپترونی طراحی و آموزش دهید که بتواند به ازای ۲۷ ویژگی ورودی در وضوح پایین، مقدار پیکسل رنگی در وضوح بالا را محاسبه کنید. معماری خود را ترسیم نموده و آموزش شبکه را توضیح دهید. از چه تابع خطایی برای آموزش استفاده کرده‌اید؟ موارد ذکر شده در ابتدای پروژه را برای این سوال به‌صورت کامل گزارش دهید و نتایج را تحلیل کنید.



\end{enumerate}