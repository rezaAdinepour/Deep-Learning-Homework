
\فصل{مقاله \مرجع{article2}}


\قسمت{ایده اصلی مقاله}
ایده اصلی این مقاله نیز، حول محور مسئله مسیریابی در طراحی‌ها با استفاده از تکنیک‌های هوش مصنوعی قرار دارد با این تفاوت که در این مقاله مسیر‌یابی ها مختص FPGA است و الگوریتم پیشنهاد شده برای مسیریابی بهتر درون FPGA پیشنهاد شده است. الگوریتم ارائه شده در این مقاله نیز متن‌باز است و مبتنی بر \texttt{Python} و کتابخانه \texttt{PyTorch} است و از موازی سازی در سطح GPU پشتیبانی می‌کند. در این مقاله ادعا شده است که طول مسیرها تا ۰٫۴ تا ۱۲٫۷ \% کوتاه شده و سرعت مسیریابی نیز بیشتر از ۲ برابر روش‌های معمولی شده است.






\قسمت{کار‌های پیشین}
همانند مقاله قبل در این مقاله نیز اشاره شده است که روش‌های قدیمی که به روش‌های CAD\پاورقی{Computer-Aided Design} معروف هستند در گذشته به طور عمده مورد استفاده قرار میگرفته است و مرحله مسیریابی به تنهایی ۴۱ تا ۸۶ درصد از کل زمان‌بندی مسیر طراحی CAD را به خود اختصاص می‌دهد \مرجع{article9} و از این رو کاری بسیار زمان‌بر است. بنابر این کیفیت و کارایی الگوریتم های PAR\پاورقی{Placement and Routing} بر طراحی بهینه بسیار تاثیر گذار هستند.


الگوریتم های بسیار زیادی مطرح شده است که همگی آنها مبتنی بر ابزار CAD ارائه شده توسط شرکت‌های سازنده FPGA هستند.

برای اینکه شرکت های تولید کننده FPGA از تولید مجدد ابزار CAD به خصوص، برای یک FPGA خاص جلوگیری کنند، ابزاری متن باز مبتنی بر کتابخانه یادگیری عمیق \texttt{PyTorch} ارائه شده است.

ایده های اصلی این مقاله را می‌توان به صورت زیر دسته بندی کرد:


\شروع{فقرات}
\فقره ابزار ارائه شده به نام \texttt{OpenPARF} است که ابزاری متن باز برای مسیریابی FPGA های پیشرفته، مبتنی بر یادگیری عمیق و قابل اجرا بر روی دو پلتفرم CPU و GPU است.
\فقره این ابزار الگوریتم های مسیریابی غیر خطی را بر اساس یک میدان الکترواستاتیکی نا متقارن پیاده سازی می‌کند که قادر است به نتایج مکانیابی برتر تحت محدودیت های مختلف مسیریابی دست پیدا کند.
\پایان{فقرات}




\قسمت{مراحل انجام الگوریتم}
مراحل انجام این الگوریتم در «شکل \رجوع{شکل:مراحل انجام الگوریتم OpenPARF}» نشان داده شده است. همانطور مشخص است، این الگوریتم دارای ۲ مرحله چینش و مسیریابی است.



\شروع{شکل}[ht]
\centerimg{img8.png}{12cm}
\شرح{مراحل انجام الگوریتم OpenPARF}
\برچسب{شکل:مراحل انجام الگوریتم OpenPARF}
\پایان{شکل}



در ابتدا فایل‌های اطلاعت مکان‌ها «فایل هایی با فرمت bookshelf» و فایل های ساختار و معماری مسیر‌یابی مورد نظر «فایل هایی با فرمت XML» در ساختار داده داخلی OpenPARF ساخته می‌شود و در مرحله دوم، بر اساس ساختار داده داخلی، OpenPARF از الگوریتم های مکان‌یابی مبتنی بر میدان‌های الکترو‌استاتیکی چندگانه مسئله مسیر‌یابی را در یک چارچوب بهینه‌سازی استاندارد استفاده می‌کند.

پس از انجام مسیریابی در مرحله اول، الگوریتم تلاش می‌کند با انجام مسیریابی های دقیق تر و با جزئیات بیشتر، مشکلات موجود در مسیر یابی اولیه انجام شده را بر طرف نماید.

پس این الگوریتم مسیر یابی از دو مرحله تشکیل می‌شود:

\شروع{فقرات}
\فقره مسیریابی اصلی سطح دانه درشت
\فقره مسیر یابی دقیق و جزئی سطح دانه ریز
\پایان{فقرات}

در مرحله اول یک مسیر اصلی و کلی پیشنهاد می‌شود و در مرحله دوم، مسیر یابی به صورت دقیق و منطقه به منطقه بررسی می‌شود و بهترین خروجی را تولید می‌کند.





\قسمت{مزایا و معایب}

از مزایای روش پیشنهاد شده می‌توان به موارد زیر اشاره کرد:
\شروع{فقرات}
\فقره این مقاله نیز همانند قبلی یکی از مزیت های آن متن باز بودن آن است
\فقره توسعه الگوریتم به دو زبان \texttt{Python} و \texttt{C++} 
\فقره توسعه الگوریتم بر دو بستر CPU و GPU
\فقره ارائه نتایج برای دو بنچ‌مارک صنعتی و اکادمیک جدید
\فقره مقایسه بسیار گسترده LUT ها و FF ها و سایر منابع مصرفی FPGA به هنگام استفاده از این الگوریتم.
\فقره و ...
\پایان{فقرات}

در کنار بیان مزایا می‌بایست به معایب پروژه را هم بیان کرد. در ادامه چند مورد از معایب پروژه انجام شده را بیان می‌کنیم:

\شروع{فقرات}
\فقره عدم اشاره مستقیم مقاله به ساختار شبکه‌عصبی استفاده شده در این الگوریتم
\فقره نیازمند بودن به سیستمی قوی برای اجرای این الگوریتم. چرا که این الگوریتم بر روی سیستمی بسیار قوی آموزش داده شده است و به این نکته اشاره نشده است که مینیمم امکانات سیستم مورد نظر چه باید باشد که بتوان این الگوریتم را بر روی آن اجرا کرد.
\فقره نیازمند بودن الگوریتم به FPGA های پیشرفته با منابع داخلی بالا. چرا که به نظر بنده این الگوریتم بر روی FPGA های قدیمی با منابع داخلی محدود قابل اجرا نمی‌باشد
\فقره عدم صحبت از توان مصرفی الگوریتم
\فقره عدم ارائه گزارش از خطا‌های ناشی از اجرای الگوریتم
\فقره و ...
\پایان{فقرات}






\قسمت{اجرای عملی الگوریتم}

\زیرقسمت{دانلود مخزن الگوریتم}

با استفاده از دستور زیر، مخزن\پاورقی{Repository} الگوریتم را دانلود می‌کنیم.
\begin{latin}
	\texttt{\textcolor{blue}{\$} git clone https://github.com/PKU-IDEA/OpenPARF.git}
\end{latin}


\زیرقسمت{نصب وابستگی‌ها پیشنیاز‌ها}
وابسگی‌های\پاورقی{Dependencies} مورد نیاز را به صورت زیر نصب می‌کنیم:
\begin{latin}
	\texttt{\# * create and activate conda environment}\\
	\texttt{\textcolor{blue}{\$} mamba create --name openparf python=3.7}\\
	\texttt{\textcolor{blue}{\$} mamba activate openparf}\\
	\texttt{\# * common packages}\\
	\texttt{\textcolor{blue}{\$} mamba install cmake boost bison}\\
	\texttt{\# * Pytorch 1.7.1. Other version may also work, but have not been tested.}\\
	\texttt{\textcolor{blue}{\$} mamba install pytorch==1.7.1 torchvision==0.8.2 cudatoolkit=11.0 -c pytorch}\\
	\texttt{\# * python packages}\\
	\texttt{\textcolor{blue}{\$} pip install hummingbird-ml pyyaml networkx tqdm}\\
\end{latin}

\قسمت{بیلد نرم‌افزار}
به صورت زیر، OpenPARF را بیلد می‌کنیم:
\begin{latin}
	\texttt{\textcolor{blue}{\$} mkdir build}\\
	\texttt{\textcolor{blue}{\$} cd build}\\
	\texttt{\textcolor{blue}{\$} cmake ../OpenPARF -DCMAKE\_PREFIX\_PATH=\$CONDA\_PREFIX}\\
	\texttt{-DPYTHON\_EXECUTABLE=\$(which python)-DPython3\_EXECUTABLE=\$(which python)}\\
	\texttt{-DCMAKE\_INSTALL\_PREFIX=<installation directory>}\\
	\texttt{\textcolor{blue}{\$} make -j8}\\
	\texttt{\textcolor{blue}{\$} make install}\\
\end{latin}

\قسمت{دانلود بنچ‌مارک ها}
بنچ‌مارک های مورد استفاده در این مقاله را می‌توان از آدرس‌های زیر دانلود نمود:

\begin{latin}
	\begin{enumerate}
		\item ISPD 2016 FPGA Placement Benchmarks [\href{https://drive.google.com/file/d/1kzg0NfEmJvwzhJADPE_Q0UjS6UpVCMZZ/view}{download}]
		\item ISPD 2016 FPGA Placement Flexshelf Benchmarks [\href{https://drive.google.com/file/d/1lwYSwfIPfzOxi_SfOZj5DyiwROLJDFd0/view}{download}]
		\item ISPD 2017 FPGA Placement Benchmarks [\href{https://drive.google.com/file/d/1Uf9qIZ8WL_jk03sIlAoS9dIrvYH3d1pz/view}{download}]
		\item ISPD 2017 FPGA Placement Flexshelf Benchmarks [\href{https://drive.google.com/file/d/1smt4lGUFdhs0TjPBzi9PqiyfA9n2Uwoy/view}{download}]
	\end{enumerate}
\end{latin}

می توان به صورت زیر نیز بنچ‌مارک ها را در ترمینال دانلود کرد:
\begin{latin}
	\texttt{\textcolor{blue}{\$} curl <url> --output <output filename>}
\end{latin}

\قسمت{لینک کردن بنچ‌مارک ها}
با استفاده از دستورات زیر، می‌توان بنچ‌مارک های دانلود شده را لینک کرد:
\begin{latin}
	\texttt{\textcolor{blue}{\$} ln -s <benchmark directory>/ispd2016 <installation}\\
	\texttt{directory>/benchmarks/ispd2016}
	
	\texttt{\textcolor{blue}{\$} ln -s <benchmark directory>/ispd2016\_flexshelf <installation}\\
	\texttt{directory>/benchmarks/ispd2016\_flexshelf}
	
	\texttt{\textcolor{blue}{\$} ln -s <benchmark directory>/ispd2017 <installation}\\
	\texttt{directory>/benchmarks/ispd2017}
	
	\texttt{\textcolor{blue}{\$} ln -s <benchmark directory>/ispd2017\_flexshelf <installation}\\
	\texttt{directory>/benchmarks/ispd2017\_flexshelf}
\end{latin}


\قسمت{اجرا کردن بنچ‌مارک ها}
بنچ‌مارک ها را به صورت زیر می‌توان اجرا\پاورقی{Run} نمود:

\begin{latin}
	\texttt{\textcolor{blue}{\$} cd <installation directory>}\\
	\texttt{\textcolor{blue}{\$} python openparf.py --config unittest/regression/ispd2016\_flexshelf/}\\
	\texttt{FPGA01\_flexshelf.json}
\end{latin}