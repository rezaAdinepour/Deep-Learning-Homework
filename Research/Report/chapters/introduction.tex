
\فصل{مقدمه}\label{فصل۱:مقدمه}

هوش مصنوعی\پانویس{Artificial Intelligence} به طور مداوم در حال تکامل است تا مشکلات سخت و پیچیده را حل کند. تولید تصویر یکی از این مشکلات پیچیده برای مدل‌های هوش مصنوعی است. \lr{GAN}‌\پانویس{Generative Adversarial Networks} ها، 
 \lr{VAE}\پانویس{Variational Autoencoder}‌ها و مدل‌های جریان عملکرد خوبی داشته‌اند اما در تولید تصاویر با وضوح بالا و دقت زیاد دچار مشکل بوده‌اند. از سوی دیگر، مدل‌های انتشار\پانویس{Diffusion Models} در تولید تصاویر با وضوح بالا و کیفیت متنوع با دقت بالا بسیار خوب عمل می‌کنند. در حال حاضر، آن‌ها در خط مقدم انقلاب هوش مصنوعی مولد (\lr{GenAI}) قرار دارند که در همه جا دیده می‌شود. مدل‌هایی مانند \lr{GLIDE}، \lr{DALL.E-3} توسط \lr{OpenAI}، \lr{Imagen} توسط گوگل، و \lr{Stable Diffusion} از جمله مدل‌های انتشار پرطرفدار هستند. در ادامه به معرفی مسئله و برخی از پیش‌نیاز ها می‌پردازیم.



%\شروع{شکل}[ht]
%\centerimg{img0}{15cm}
%\شرح{ساختار کلی حافظه‌ها}
%\برچسب{شکل:ساختار حافظه‌ها}
%\پایان{شکل}



\قسمت{تعریف مسئله}


\زیرقسمت{انتشار چیست؟}
انتشار یک پدیده طبیعی اساسی است که در سیستم‌های مختلف از جمله فیزیکی، شیمیایی و زیست‌شناسی مشاهده می‌شود.

این پدیده در زندگی روزمره به وضوح قابل مشاهده است. به عنوان مثال، در نظر بگیرید که عطر را اسپری می‌کنید. در ابتدا، مولکول‌های عطر به طور متراکم در نزدیکی نقطه اسپری قرار دارند. با گذشت زمان، مولکول‌ها در محیط اطراف منتشر می‌شوند.

انتشار فرآیندی است که طی آن ذرات، اطلاعات یا انرژی از ناحیه‌ای با غلظت بالا به ناحیه‌ای با غلظت پایین‌تر حرکت می‌کنند. این اتفاق به این دلیل رخ می‌دهد که سیستم‌ها تمایل دارند به تعادل\پانویس{Equilibrium} برسند، جایی که غلظت‌ها در سراسر سیستم یکسان می‌شود.

در یادگیری ماشین\پانویس{Machine Learning} و تولید داده\پانویس{Data Generation}، انتشار به یک رویکرد خاص برای تولید داده‌ها با استفاده از یک فرآیند تصادفی مشابه با زنجیره مارکوف\پانویس{Markov Chain} اشاره دارد. در این زمینه، مدل‌های انتشار نمونه‌های جدیدی از داده‌ها را با استفاده از داده‌های ساده‌تر ایجاد می‌کنند و به تدریج داده‌های پیچیده‌تر و واقعی‌تر تولید می‌شود.







\زیرقسمت{مدل‌های انتشار در یادگیری ماشین چیستند؟}
مدل‌های انتشار مولد\پانویس{Generative} هستند، به این معنی که داده‌های جدیدی بر اساس داده‌هایی که بر روی آن‌ها آموزش دیده‌اند تولید می‌کنند. به عنوان مثال، یک مدل انتشار که بر روی مجموعه‌ای از داده‌های چهره‌های انسان آموزش دیده است، می‌تواند چهره‌های انسانی جدید و واقع‌گرایانه با ویژگی‌ها و حالت‌های مختلف تولید کند، حتی اگر آن چهره‌های خاص در مجموعه داده‌های اولیه وجود نداشته باشند.

برخلاف سایر مدل‌های مولدی مثل \lr{GAN}، \lr{VAE} و ... این مدل‌ بر مدل‌سازی تکامل مرحله به مرحله توزیع داده‌ها از یک نقطه شروع ساده به یک توزیع پیچیده‌تر تمرکز دارند. مفهوم اساسی مدل‌های انتشار این است که یک توزیع ساده مثل توزیع گاوسی\پانویس{Gaussian Distribution}، را از طریق یک سری عملیات های قابل برگشت\پانویس{Invertible Operations} به یک توزیع داده پیچیده‌تر تبدیل کنند.

\شروع{شکل}[ht]
\centerimg{img0}{15cm}
\شرح{ساختار مدل‌های مولد}
\برچسب{شکل:ساختار مدل‌های مولد}
\پایان{شکل}

پس از اینکه مدل، فرآیند تبدیل را آموزش می‌بیند، می‌تواند نمونه‌های جدیدی را با شروع از یک نقطه در توزیع ساده و به تدریج "انتشار" آن به توزیع داده پیچیده مطلوب تولید کند.

در مدل‌های \lr{DDPMs}\پانویس{Denoising Diffusion Probabilistic Models}
با افزودن تدریجی نویز گاوسی به داده‌های اصلی در فرآیند انتشار پیشرو\پانویس{Forward Diffusion} و سپس یادگیری حذف نویز در فرآیند انتشار معکوس\پانویس{Reverse Diffusion} کار می‌کنند. «شکل \رجوع{شکل:نحوه عملکرد مدل‌های DDPMs}»


\شروع{شکل}[ht]
\centerimg{img1}{15cm}
\شرح{نحوه عملکرد مدل های \lr{DDPMs} \مرجع{ho2020denoising}}
\برچسب{شکل:نحوه عملکرد مدل‌های DDPMs}
\پایان{شکل}




\قسمت{اهمیت پژوهش}
مدل‌های انتشار با شبیه‌سازی فرآیندهای تصادفی قادر به تولید نمونه‌های واقعی‌تری هستند که در کاربردهای مختلفی مانند ترمیم تصاویر، نویززدایی و تولید تصاویر با وضوح بالا استفاده می‌شوند. به ویژه، ترانسفورمرهای انتشار (\lr{DiT}) به عنوان یک نوآوری در این حوزه با استفاده از معماری ترانسفورمر، قدرت بیشتری در مدل‌سازی پیچیدگی‌های داده‌ها دارند. \lr{DiT}ها با توانایی مقیاس‌پذیری بالا و کاهش مداوم معیار \lr{FID}، بهبود قابل توجهی در کیفیت و واقع‌گرایی نمونه‌های تولیدی ارائه می‌دهند. این ویژگی‌ها \lr{DiT}ها را به ابزار قدرتمندی برای وظایف پیچیده‌تری مانند سنتز تصویر و بهبود کیفیت تصویر تبدیل می‌کند.


\قسمت{ساختار پژوهش}
اینن پژوهش در ۵ فصل انجام شده است. در فصل \ref{فصل۱:مقدمه} به مقدمه و اهمیت موضوع پژوهش پرداخته شده است. در فصل \ref{فصل۲:مفاهیم اولیه} به مفاهیم اولیه و پیش‌نیاز ها پرداخته شده است. در ادامه در فصل سوم پژوهش به بررسی کار‌های پیشین انجام شده در این زمینه پرداخت شده است. در فصل چهارم به بررسی دقیق و جزئی مقالات مطالعه شده در این پژوهش پرداخته شده است و در فصل پایانی، جمع‌بندی و نتیجه گیری پژوهش ارائه شده است.






























%
%مراحل طراحی IC را می‌توان به ۴ مرحله زیر تقسیم کرد:
%\شروع{شمارش}
%\فقره طراحی شماتیک بخش‌های مختلف مدار
%\فقره آنالیز و بررسی طراحی انجام شده و اطمینان از صحت عملکرد مدار به وسیله نرم‌افزار های شبیه‌سازی مانند SPICE و Cadence
%\فقره مرحله Layout که شامل Placement و Routing است
%\فقره ساخت آیسی یا Fabrication
%\پایان{شمارش}
%
%مرحله ۱ و ۲ باید به صورت دستی و توسط انسان انجام شود. چرا که شخص طراح می‌بایست بر همه بخش‌های طراحی خود مسلط باشد و بتواند اگر نیاز بود بخش‌های دیگیری به طراحی اضافه و یا از آن کم شود، آن را اعمال کند. اما هوش مصنوعی به این مرحله نیز وارد شده است و ابزار‌هایی ماندد Magic EDA\زیرنویس{برای اطلاعات بیشتر می‌توان به اینجا مراجعه کرد: \href{www.snapmagic.com}{snapmagic.com}} مخصوص این کار آموزش داده شده است که با دادن اطلاعات مورد نیاز خود برای طراحی، مدار مورد نیاز ما را به صورت کامل طراحی می‌کند. که در این گزارش به آن نمی‌پردازیم.
%
%
%در مرحله ساخت آیسی\پاورقی{Fabrication} نیز هوش‌مصنوعی به صورت محدود وارد شده است و همچنان مرحله ساخت به صورت قدیمی و سنتی انجام می‌شود.
%
%
%در گذشته، در مرحله ۳، طراحی‌ ها با استفاده از ابزار‌های کامپیوتری CAD\پاورقی{Computer Aided Design} انجام می‌شود. از مزایا ابزار‌های CAD می‌توان به موارد زیر اشاره کرد:
%
%\شروع{فقرات}
%\فقره تحلیل‌ دقیق
%\فقره تولید خروجی باکیفیت
%\فقره پشتیبانی گسترده نرم‌افزاری
%\پایان{فقرات}
%
%اما در کنار مزایای نامبرده می‌توان به معایب آن هم اشاره کرد:
%\شروع{فقرات}
%\فقره لزوم وجود کاربر انسانی\پاورقی{Designer} برای انجام طراحی
%\فقره زمان زیاد برای انجام
%\فقره هزینه بسیار بالای ابزار‌های CAD
%\پایان{فقرات}
%
%
%با پیشرفت ابزار‌های هوش‌مصنوعی مانند شبیه‌های عصبی\پاورقی{Neural Network} ابزار‌های مختلفی که برپایه شبکه‌های عصبی کار می‌کنند معرفی شده است.  این ابزار‌ها با حذف اپراتور انسانی در فرایند Place\&Route و کاهش زمان انجام این فاز از طراحی، کمک بزرگی به این زمینه کرده است.
%
%
%
%\قسمت{اهمیت موضوع}
%در طراحی‌های تجاری، طراح‌ها مجبور‌اند چندین بار طراحی خود را برای دست‌یابی به بهترین و بهینه\پاورقی{Optimum} ترین حالت عوض کنند. استفاده از روش‌های طراحی‌ سنتی قدیمی، برای مدار‌های بزرگ\پاورقی{Complex} امروزی، بسیار فرایندی طولانی و کند است که فرایند تکرار طراحی برای دستیابی به بهینه‌ترین حالت جایگذاری و سیم‌کشی را به شدت کند می‌کند.
%
%
%
%
%
%\قسمت{اهداف پژوهش}
%در اصل، در این مقالات، یک مسئله بهینه‌سازی غیر خطی حل شده است و به مسئله جایگذاری و سیم‌کشی به عنوان یک فرایند غیرخطی نگاه شده است که قرار است آن را بهینه کنیمو به طوری اهداف ما یعنی پیدا کردن بهترین محل قرار گیری سلول‌های طراحی با حداقل سیم‌کشی ممکن که کمترین همپوشانی را داشته باشد ارضا شود.



%
%
%در این پروژه هدف طراحی و شبیه‌سازی file Register ای با اندازه ۱۲۸ کلمه ۳۲ بیت است. هدف از انجام این پروژه آشنایی و انواع حافظه‌ها و نحوه شبیه‌سازی و پیاده‌سازی آنهاست. حافظه‌ها معمولا از دو بخش تشکیل می‌شوند:
%\صفحه‌جدید
%\شروع{فقرات}
%\فقره بخش حافظه
%\فقره بخش سخت‌افزار
%\پایان{فقرات}
%
%بخش حافظه مبتنی‌ست بر تکرار یک طراحی مشخص از یک سلول\پاورقی{Cell} حافظه با چینشی مشخص. بخش سخت افزار آن متشکل است از دیکدر\پاورقی{Decoder} آدرس و مالتی‌پلکسر\پاورقی{Multiplaxer} داده ها.
%
%
%
%\قسمت{مراحل انجام پروژه}
%
%برای اطمینان از انجام پروژه و مقایسه بین مدل  RTL و مدل سطح ترانزیستوری، این پروژه به دو بخش تقسیم شده است.
%
%\زیرقسمت{فاز اول}
%در فاز اول با استفاده از زبان توصیف سخت‌افزار\پاورقی{Hardware describtion language} VHDL کد RTL حافظه SRAM نوشته و شبیه‌سازی شده است.
%
%برای مقایسه بین دو مدل RTL و رفتاری\پاورقی{Behavioral} حافظه، یک کد مجزا هم برای شبیه‌سازی مدل رفتاری نوشته شده است.
%
%
%\زیرقسمت{فاز دوم}
%در فاز دوم پروژه مقیاس طراحی را به سطح ترانزیستور می‌آوریم و به کمک نرم‌افزار HSpice مدل شبیه‌سازی شده در فاز اول را در سطح ترانزیستور شبیه‌سازی می‌کنیم.
%
%\مهم{*** به دلیل عدم تکمیل فاز دوم پروژه در این ددلاین، این بخش از گزارش پس از تکمیل فاز دو تکمیل خواهد شد.***}
