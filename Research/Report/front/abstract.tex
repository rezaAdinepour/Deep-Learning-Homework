% -------------------------------------------------------
%  Abstract
% -------------------------------------------------------


\شروع{وسط‌چین}
\مهم{چکیده}
\پایان{وسط‌چین}
\بدون‌تورفتگی

چالش اصلی این تحقیق، بررسی و مقایسه کارایی ترنسفرمرها در برابر معماری‌های سنتی مانند U-Net در مدل‌های انتشار بوده است. نتایج به دست آمده نشان می‌دهند که ترنسفرمرها با مقیاس‌پذیری بهتر و کارایی بالاتر می‌توانند جایگزین مناسبی برای معماری‌های کنونی در مدل‌های انتشار باشند و به بهبود کیفیت تصاویر تولید شده کمک کنند. در این راستا، ترنسفرمرهای انتشار (DiTs) با افزایش عمق و عرض ترنسفرمرها و افزایش تعداد توکن‌های ورودی، توانسته‌اند بهبود قابل توجهی در معیار FID نشان دهند و نتایج برتری در بسته‌های محک ImageNet به دست آورند.

این تحقیق نشان می‌دهند که استفاده از ترنسفرمرها در مدل‌های انتشار می‌تواند راهکارهای نوینی برای بهبود کیفیت و کارایی در تولید تصاویر ارائه دهد. این رویکردها، علاوه بر بهبود عملکرد، می‌توانند به کاهش پیچیدگی‌های محاسباتی و افزایش سرعت فرآیندهای تولید تصویر کمک کنند. با توجه به این نتایج، ترنسفرمرها به عنوان یک جایگزین قوی برای معماری‌های سنتی در مدل‌های انتشار مطرح می‌شوند و می‌توانند به طور گسترده در کاربردهای مختلف مورد استفاده قرار گیرند.




\پرش‌بلند
\بدون‌تورفتگی \مهم{کلیدواژه‌ها}: 
یادگیری عمیق، مدل‌های انتشار، ترنسفرمر، یو-نت، کدگذار، کدگشا
\صفحه‌جدید
